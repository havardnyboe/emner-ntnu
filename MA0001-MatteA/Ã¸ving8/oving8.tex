% #region PREAMBEL OG PAKKER
\documentclass[a4paper, 12pt]{article}  % DOKUMENTKLASSE
\title{Øving 8}               % TITTEL
\author{Håvard Solberg Nybøe}           % FORFATTER
\date{MA0001 -- \today}                 % DATO & FAG

\usepackage[english, norsk]{babel}      % NORSK SPRÅK
\usepackage[
    backend=biber,style=apa]{biblatex}  % BIBLIOGRAFI
\usepackage{csquotes}                   % PAKKE TIL BABEL
\addbibresource{bibliografi.bib}        % PATH TIL BIBLIOGRAFI
\usepackage[hidelinks]{hyperref}        % LENKER I TOC OG GENERELT
\usepackage[margin=1in]{geometry}       % VANLIG STØRRELSE MARGIN
\setlength{\parindent}{0em}             % SKILLER AVSNITT
\setlength{\parskip}{.8em}              % SKILLER AVSNITT
\usepackage{graphicx}                   % BILDER \includegraphics[OPTIONS]{PATH}
\usepackage{kantlipsum}                 % FYLLTEKST I KANT-STIL (kant[n-m])
\usepackage{amsfonts,                   % BLACKBOARD BOLD FONT (\mathbb{N})
amsmath,stmaryrd,amssymb}               % ANDRE MATTE PAKKER
\usepackage{circuitikz, pgfplots}       % LOGISKE PORTER OG KRETSER & TikZ
\usepackage{import}                     % IMPORTER FILER (\import{PATH}{FILE})
\usepackage{caption}                    % PAKKE FOR BEDRE CAPTIONS I FIGURER
\usepackage{float}                      % TVINGER PLASSERING AV FIGURER
% #endregion
\begin{document}

\maketitle
% \tableofcontents % INNHOLDSFORTEGNELSE

\begin{enumerate}
    \item [\boxed{1}]
          \begin{flalign*}
              y^2 + y + x^4 + 3x - 4               & = 0  & \\
              \frac{d}{dx} (y^2 + y + x^4 + 3x - 4 & = 0) & \\
              \Downarrow                           &        \\
              4x^3 + 3                             & = 0
          \end{flalign*}
          Tangenten til $y$ i $(1, -1)$:
          \begin{flalign*}
              y & = 7x - 8 &
          \end{flalign*}
    \item [\boxed{2}] Bruker L'Hôspitals regel til å regne ut grenseverdiene.
          \begin{enumerate}
              \item $\displaystyle \lim_{x \to 1} \frac{\sin(\pi x)}{\ln(x^{2\pi})} = -{1\over 2}$
              \item $\displaystyle \lim_{x \to +\infty} x^3 \cdot 3^{-3} = 0$
              \item $\displaystyle \lim_{x \to \infty} \left( 1 + {1 \over x} \right)^x = e$
          \end{enumerate}
    \item [\boxed{3}]
          \begin{enumerate}
              \item $\displaystyle g(x) = x^{1\over3} \Rightarrow {5\over20736}x³ - {1\over108}x² + {5\over27}x + {80\over81}, \quad\text{Taylorpolynom om } x_0 = 8$
              \item $\displaystyle h(x) = \sin(e^x) \Rightarrow $
          \end{enumerate}
          \newpage
    \item [\boxed{4}] \textbf{ }
          \begin{figure}[H]
              \begin{center}
                  \begin{tikzpicture}
                      \begin{axis}[xmin=-5, xmax=6, ymin=-4, ymax=8, axis on top, style=thick, grid, grid style={dashed,gray!30}, axis equal image]
                          \addlegendentry{$f(x)=x^3-3x+2$};
                          \addplot[color=red, smooth]{x^3-3*x+2};
                          \draw [black, fill] (axis cs: -1,4) circle (2pt) node[anchor=west] {$(-1,4)$};
                          \draw [black, fill] (axis cs: 1,0) circle (2pt) node[anchor=west] {$(1,0)$};
                      \end{axis}
                  \end{tikzpicture}
              \end{center}
          \end{figure}
          Toppunkt: $(-1, 4)$, bunntpunkt: $(1, 0)$

          Funksjonen vokser i intervallet $x \in [-\infty, -1]$ og $x \in [1, \infty]$.
          Funksjonen synker i intervallet $x \in [-1, 1]$.
\end{enumerate}

% \printbibliography[heading=bibintoc] % LAGER BIBLIOGRAFI
\end{document}
