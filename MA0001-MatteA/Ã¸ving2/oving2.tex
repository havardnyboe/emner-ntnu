% #region PREAMBEL OG PAKKER
\documentclass[a4paper, 12pt]{article}  % DOKUMENTKLASSE
\title{Øving 2}               % TITTEL
\author{Håvard Solberg Nybøe}           % FORFATTER
\date{MA0001 -- \today}                    % DATO & FAG

\usepackage[english, norsk]{babel}      % NORSK SPRÅK
\usepackage[
    backend=biber,style=apa]{biblatex}  % BIBLIOGRAFI
\usepackage{csquotes}                   % PAKKE TIL BABEL
\addbibresource{bibliografi.bib}        % PATH TIL BIBLIOGRAFI
\usepackage[hidelinks]{hyperref}        % LENKER I TOC OG GENERELT
\usepackage[margin=1in]{geometry}       % VANLIG STØRRELSE MARGIN
\setlength{\parindent}{0em}             % SKILLER AVSNITT
\setlength{\parskip}{.8em}              % SKILLER AVSNITT
\usepackage{graphicx}                   % BILDER \includegraphics[OPTIONS]{PATH}
\usepackage{kantlipsum}                 % FYLLTEKST I KANT-STIL (kant[n-m])
\usepackage{amsfonts,                   % BLACKBOARD BOLD FONT (\mathbb{N})
amsmath,stmaryrd,amssymb}               % ANDRE MATTE PAKKER
\usepackage{circuitikz}                 % LOGISKE PORTER OG KRETSER & TikZ
\usepackage{import}                     % IMPORTER FILER (\import{PATH}{FILE})
\usepackage{xcolor}                     % FARGE
\usepackage{float}                      % TVINGER PLASSERING AV FIGURER
\usepackage{pgfplots}                   % PGF PLOTTING
% #endregion
\begin{document}

\maketitle
% \tableofcontents % INNHOLDSFORTEGNELSE

\begin{enumerate}
    \item [\boxed{1}]
          \begin{enumerate}
              \item [\textbf{a)}]
                    \begin{flalign*}
                        \left|5-2x\right| < 3 &\Leftrightarrow -3 < 5 - 2x < 3 &&\\
                        & \Leftrightarrow -(-3 -5) < 2x < -(3-5) &&\\
                        & \Leftrightarrow 8 > 2x > 2 &&\\
                        & \Leftrightarrow 4 < x < 1 &&\\
                        & \boxed{x \in \left[1,4\right]}
                    \end{flalign*}
              \item [\textbf{b)}]
                    \begin{flalign*}
                        \left|x^2-3\right| < 6 & \Leftrightarrow -6 < x^2 -3 < 6 &&\\
                        & \Leftrightarrow -6 + 3 < x^2 < 6 + 3 &&\\
                        & \Leftrightarrow -3 < x < 3 &&\\
                        & \boxed{x \in [-3, 3]}
                    \end{flalign*}
          \end{enumerate}
    \item [\boxed{2}]
          \begin{enumerate}
              \item [\textbf{a)}]
                    \begin{flalign*}
                        \left| -4.5 - 1.1\right| &= \left|-4.5 - c_1\right| + \left|c_1 - 1.1\right| &&\\
                        5.6 - (\left|-4.5 - c_1\right|) &= \left|c_1 - 1.1\right| &&\\
                        &\Updownarrow &&\\
                        5.6 + 4.5 - c_1 = c_1 - 1.1 &\lor 5.6 -4.5 - c_1 = c_1 - 1.1 &&\\
                        2c_1 = 11.2 &\lor 2c_1 = 0 &&\\
                        \boxed{c_1 = 5.6} &\lor \boxed{c_1 = 0} &&\\
                    \end{flalign*}
                    \begin{tikzpicture}[thick]
                        \draw(-4,0)--(8,0);
                        \foreach \x/\xtext in {-4,..., 8}
                        \draw(\x,0pt)--(\x,3pt) node[above] {\xtext};
                        \foreach \x in {-4,..., 8}
                        \draw(\x,0pt)--(\x,-3pt) node[below] {};
                        \draw(0,-3pt) node[below] {\color{red}$c_1$};
                        \draw(0,3pt) node[above] {\color{red}$0$};
                        \draw(5.6,0)--(5.6,-3pt) node[below] {\color{red}$c_1$};
                        \draw(5.6,0)--(5.6,3pt) node[above] {\color{red}$5.6$};
                    \end{tikzpicture}
              \item [\textbf{b)}]
                    \begin{flalign*}
                        \left| -4.5 - 1.1 \right| &< \left| -4.5 - c_2 \right| + \left| c_2 - 1.1 \right| &&\\
                        5.6 &< \left| -4.5 - c_2 \right| + \left| c_2 - 1.1 \right| &&\\
                        &\Updownarrow &&\\
                        -5.6 > -4.5 - c_2 + c_2 - 1.1 &\lor 5.6 < -4.5 - c_2 + c_2 - 1.1 &&\\
                    \end{flalign*}
          \end{enumerate}
    \item [\boxed{3}]
          \begin{flalign*}
              a, b \in \mathbb{R}     & , \quad a \neq b                     && \\
              f(x)                    & : \mathbb{R}  \rightarrow \mathbb{R} && \\
              f(a)= b           \quad & \land \quad f(b)= a                  && \\
              f(x) = {1\over a} x + b \quad & \land \quad f(x) = {1\over b} x + a && \\
              f(x) &= \left({1\over a} x + b\right) + \left({1\over b} x + a\right) &&\\
              f(x) &= {x\over a} + {x\over b} + b + a &&\\
              f(x) &= \frac{x (a+b) + ab(a + b)}{ab} &&\\
              f(x) &= x (a+b) + (a + b)
          \end{flalign*}
    \item [\boxed{4}] La $y_1$ og $y_2$ være to linjer med likningene:
          \begin{align*}
              y_1 & = a_1 x + b_1 \\
              y_2 & = a_2 x + b_2
          \end{align*}
          $y_1$ og $y_2$ er parallelle hvis $a_1 \cdot a_2 = -1$
          \begin{align*}
              5x+3y                  & =-4                                           \\
                                     & \Downarrow                                    \\
              y_1                    & = -\frac{5}{3}x - \frac{4}{3}                 \\
              -\frac{5}{3} \cdot a_2 & = -1                                          \\
              a_2                    & = \frac{3}{5}                                 \\
              b_2                    & = 4, \textrm{ siden $y_2$ skjærer i $(0,4)$ } \\
              y_2                    & = \frac{3}{5}x + 4                            \\
          \end{align*}
          \begin{figure}[H]
              \begin{center}
                  \begin{tikzpicture}
                      \begin{axis}[xmin=-5, xmax=6, ymin=-4, ymax=8, axis on top, style=thick, grid, grid style={dashed,gray!30}, axis equal image]
                          \addplot[color=blue]{-(5/3)*x-4/3};
                          \addlegendentry{$y_1=-{5\over3}x-{4\over3}$};
                          \addplot[color=red]{(3/5)*x+4};
                          \addlegendentry{$y_2=\frac{3}{5}x + 4$};
                          \draw [black, fill] (axis cs: 0,4) circle (2pt) node[anchor=west] {$(0,4)$};
                      \end{axis}
                  \end{tikzpicture}
              \end{center}
          \end{figure}
\end{enumerate}

% \printbibliography[heading=bibintoc] % LAGER BIBLIOGRAFI
\end{document}