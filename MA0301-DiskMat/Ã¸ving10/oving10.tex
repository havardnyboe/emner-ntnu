% #region PREAMBEL OG PAKKER
\documentclass[a4paper, 12pt]{article}  % DOKUMENTKLASSE
\title{Øving 10}                        % TITTEL
\author{Håvard Solberg Nybøe}           % FORFATTER
\date{MA0301 -- \today}                 % DATO & FAG

\usepackage[english, norsk]{babel}      % NORSK SPRÅK
\usepackage[
    backend=biber,style=apa]{biblatex}  % BIBLIOGRAFI
\usepackage{csquotes}                   % PAKKE TIL BABEL
\addbibresource{bibliografi.bib}         % PATH TIL BIBLIOGRAFI
\usepackage[hidelinks]{hyperref}        % LENKER I TOC OG GENERELT
\usepackage[margin=1in]{geometry}       % VANLIG STØRRELSE MARGIN
\setlength{\parindent}{0em}             % SKILLER AVSNITT
\setlength{\parskip}{.8em}              % SKILLER AVSNITT
\usepackage{graphicx}                   % BILDER \includegraphics[OPTIONS]{PATH}
\usepackage{kantlipsum}                 % FYLLTEKST I KANT-STIL (kant[n-m])
\usepackage{amsfonts,                   % BLACKBOARD BOLD FONT (\mathbb{N})
amsmath,stmaryrd,amssymb,amsthm}        % ANDRE MATTE PAKKER
\usepackage{circuitikz}                 % LOGISKE PORTER OG KRETSER & TikZ
\usepackage{import}                     % IMPORTER FILER (\import{PATH}{FILE})
\usepackage{caption}                    % PAKKE FOR BEDRE CAPTIONS I FIGURER
\usepackage{float}                       % FLYTT FIGURER 
% #endregion
\begin{document}

\maketitle
% \tableofcontents % INNHOLDSFORTEGNELSE

\begin{enumerate}
    \item [\boxed{1}]
    \begin{enumerate}
        \item Hver av mattematikerene kan velges med en av informatikerene. 
        \newline
        \begin{center}
            \begin{tabular}{ccccc}
                       & $m_1$ & $m_2$ & $m_3$ & $\cdots$\\
                $cs_1$ & $(cs_1, m_1)$ & $(cs_1, m_2)$ & $(cs_1, m_3)$ & $\cdots$\\
                $cs_2$ & $(cs_2, m_1)$ & $(cs_2, m_2)$ & $(cs_2, m_3)$ & $\cdots$\\
                $cs_3$ & $(cs_3, m_1)$ & $(cs_3, m_2)$ & $(cs_3, m_3)$ & $\cdots$\\
                $\vdots$ & $\vdots$ & $\vdots$ & $\vdots$ & \\
            \end{tabular}
            \\[1em]
            \(18 \cdot 325 = \boxed{5850}\)
        \end{center}
        \item Hvis den man skal velge kan være i en av de to kategorene, vil svaret bli summen av kategoriene.
        \begin{align*}
            \frac{(18+325)!}{(18+325-1)!} = \boxed{343} = 18 + 325
        \end{align*}
    \end{enumerate}
    \item [\boxed{2}] Hvis strengen starter og slutter på 1, vil antall 10-bit strenger være lik antall 8-bit strenger.
    \begin{align*}
        2^8 = \boxed{256}
    \end{align*}
    \item [\boxed{3}] 
    \begin{enumerate}
        \item Hvis man rullerer de 26 bokstavene 26 ganger vil man havne tilbake til utgangspunktet, bokstavene kan derfor bare rulleres 25 ganger. Ergo finnes det kun 25 Cæsarchiffer.
        \item Anta at \(a\) <<velger>> en bokstav, \(b\) kan da <<velge>> fra de gjennværende 25 osv. 
        \begin{align*}
            \sum_{n=26}^{1} = 26 + 25 + 24 + \cdots + 1 = \boxed{351}
        \end{align*}
        \item \text{}
        \begin{align*}
            \sum_{n=21}^{1} = 231
            \sum_{n=5}^{1} = 15
            231 + 15 = \boxed{246}
        \end{align*}
    \end{enumerate}
    \newpage
    \item [\boxed{4}]
    \begin{enumerate}
        \item \(\displaystyle P(6,3) = \frac{6!}{(6-3)!} = 120\)
        \item \(\displaystyle P(6,5) = \frac{6!}{(6-5)!} = 720\)
        \item \(\displaystyle P(8,1) = \frac{8!}{(8-1)!} = 8\)
        \item \(\displaystyle P(8,5) = \frac{8!}{(8-5)!} = 6720\)
        \item \(\displaystyle P(8,8) = \frac{8!}{(8-8)!} = 40320\)
        \item \(\displaystyle P(10,9) = \frac{10!}{(10-9)!} = 3628800\)
    \end{enumerate}
    \item [\boxed{5}]
    \begin{enumerate}
        \item \(\displaystyle C(5,1) = \binom{5}{1} = 5\)
        \item \(\displaystyle C(5,3) = \binom{5}{3} = 10\)
        \item \(\displaystyle C(8,4) = \binom{8}{4} = 70\)
        \item \(\displaystyle C(8,8) = \binom{8}{8} = 1\)
        \item \(\displaystyle C(8,0) = \binom{8}{0} = 1\)
        \item \(\displaystyle C(12,6) = \binom{12}{6} = 924\)
        \end{enumerate}
    \item [\boxed{6}] \(\displaystyle \binom{26}{5} = \frac{26!}{5! \cdot 21!} = 65780\)
    \item [\boxed{7}] 
    \begin{flalign*}
        \binom{5}{3} = \frac{5!}{3! \cdot 2!} &= 10 &&\\ 
        3! \cdot 10 &= 60   
    \end{flalign*}
    \item [\boxed{8}]
    \begin{enumerate}
        \item \(\displaystyle \binom{9}{4} = \frac{9!}{4! \cdot 5!} = 126\)
        \item \(\displaystyle \binom{9}{5} = \frac{9!}{5! \cdot 4!} = 126\)
    \end{enumerate}
    \item [\boxed{9}] 
    \begin{flalign*}
        \binom{r+b}{2} = \frac{(r+b)!}{2! \cdot (r+b-2)!} \\
        \binom{r}{2} = \frac{r!}{2! \cdot (r-2)!} \\
        \binom{b}{2} = \frac{b!}{2! \cdot (b-2)!} \\
    \end{flalign*}
    \item [\boxed{10}]
    \begin{enumerate}
        \item \(\displaystyle \binom{31}{12} = \frac{31!}{12! \cdot 19!} = 141 120 525\)
        \item \(\displaystyle \binom{31+12-1}{12} = \frac{(31+12-1)!}{12! \cdot (31+12-1-12)!} = 11058116888\)
    \end{enumerate}
\end{enumerate}

% \printbibliography[heading=bibintoc] % LAGER BIBLIOGRAFI
\end{document}
