% #region PREAMBEL OG PAKKER
\documentclass[a4paper, 12pt]{article}  % DOKUMENTKLASSE
\title{Øving 5}                         % TITTEL
\author{Håvard Solberg Nybøe}           % FORFATTER
\date{MA0301 -- \today}                 % DATO & FAG

\usepackage[english, norsk]{babel}      % NORSK SPRÅK
\usepackage[
    backend=biber,style=apa]{biblatex}  % BIBLIOGRAFI
\usepackage{csquotes}                   % PAKKE TIL BABEL
\addbibresource{bibliografi.bib}         % PATH TIL BIBLIOGRAFI
\usepackage[hidelinks]{hyperref}        % LENKER I TOC OG GENERELT
\usepackage[margin=24mm]{geometry}       % VANLIG STØRRELSE MARGIN
\setlength{\parindent}{0em}             % SKILLER AVSNITT
\setlength{\parskip}{.8em}              % SKILLER AVSNITT
\usepackage{graphicx}                   % BILDER \includegraphics[OPTIONS]{PATH}
\usepackage{kantlipsum}                 % FYLLTEKST I KANT-STIL (kant[n-m])
\usepackage{amsfonts,                   % BLACKBOARD BOLD FONT (\mathbb{N})
amsmath,stmaryrd,amssymb}               % ANDRE MATTE PAKKER
\usepackage{circuitikz}                 % LOGISKE PORTER OG KRETSER & TikZ
\usepackage{import}                     % IMPORTER FILER (\import{PATH}{FILE})
\usepackage{caption}                    % PAKKE FOR BEDRE CAPTIONS I FIGURER
\usepackage{float}                       % FLYTT FIGURER 
% #endregion
\begin{document}

\maketitle
% \tableofcontents % INNHOLDSFORTEGNELSE

\textit{Ønsker retting}

\begin{enumerate}
    \item [\boxed{1}] \(R = \{(m,n) \in \mathbb{N}\times\mathbb{N}|\exists l \in \mathbb{N}:n=lm\}\)
    \\[1em]\(R\) er refleksiv fordi \(R(m,n)=R(n,m)\)
    \\\(R\) er antisymmetrisk fordi \(R(m,n)=R(n,m)\)
    \\\(R\) er transitiv fordi \(R(m,n)=R(R(m,n),R(n,m))\) 
    \\[1em]\(R\) er derfor en delvis ordnet mengde. \(\square\)

    \item [\boxed{2}] 
    \begin{enumerate}
        \item \textit{1.} er en funksjon fordi for alle \(x\) under relasjonen \(R\) har den bare ett bilde \(y\). (alle \(x\)-verdier svarer til én \(y\)-verdi).
        \\\textit{2.} er en funksjon fordi for alle \(x\) under relasjonen \(R\) har den bare ett bilde \(y\).
        \item \textit{1.} er bijektiv fordi for alle elementene i verdimengden er bare avbildet en gang av et element i definisjonsmengden.
        \\\textit{2.} er surjektiv fordi for alle elementene i verdimengden er avbildet minst en gang av et element i definisjonsmengden \((\sin(0) = \sin(2\pi) = 0)\).
    \end{enumerate}
    
    \item [\boxed{3}] \(f_R = \{(x,y) \in X \times (X/R)|x \in y\}\)
    \begin{enumerate}
        \item [(b)] den inverse funksjonen til \(f_R\) er \(f_R^{-1} = \{(y,x) \in X \times (X/R)|x \in y\}\)
    \end{enumerate}
    
    \item [\boxed{4}]
    \begin{enumerate}
        \item Fra posisjonen \((1, 1)\) kan springeren flytte til \((2, 3)\) får så å komme til \((3, 1)\).
        \\\(((1,1),(3,1)) \in R\) \(\square\)
        \item Gitt observasjonen i (a) er \(R\) transitiv. 
        \\Relasjonen er symetrisk fordi hvis en springer kan flytte fra \((1,1)\) til \((2,3)\) så kan den også flytte fra \((2,3)\) til \((1,1)\).
        \\Relasjonen er refleksiv fordi \(((1,1),(1,1)) \in R\), (springeren kan ``flytte'' til feltet den står på).
        \\Relasjonen en derfor ekvivalensrelasjon \(\square\)
    \end{enumerate}
    
    \item [\boxed{5}]
    Gitt at \(f\) og \(f^{-1}\) er funksjoner, må de i det minstre være injektive for at \(f^{-1}\) skal være inversfunksjon av \(f\).
    \(f^{-1}\) er definert slik at \(f^{-1} = \{(y,x)|y=f(x)\}\), siden \(y=f(x)\) tilsvarer en hver y i definisjonsmengden en x i verdimengden. 
    \(f^{-1}\) surjektiv og dermed også bijektiv.
    
    \item [\boxed{6}] 
    \begin{enumerate}
        \item \(f:\mathbb{Z} \rightarrow \mathbb{Z}, f(n)=2n\), 
        \\ikke injektiv fordi \(f(-1)=f(1)=1\), surjektiv fordi for alle elementene i verdimengden finnes det minst ett element i definisjonsmengden som relaterer til dette elementet med funksjonen \(f\).
        \item \(f:\mathbb{R}\rightarrow\{x\in\mathbb{R}|0\leqslant x < 1\}, f(x) = x - \lfloor x\rfloor \),
        \\
        \item \(f:\mathbb{N}\times\mathbb{N}\rightarrow\mathbb{N}, \text{ hvor } f(n, m) \text{ er den største av } m \text{ og } n\),
        \\ikke injektiv da både f.eks. (1,1) og (0,1) svarer til 1, surjektiv da det for alle elementene i verdimengden finnes minst ett element i definisjonsmengden som relaterer til elementet med funksjonen \(f\).
        \item \(f:\mathbb{Z}\rightarrow\mathbb{R}, f(n)={n\over 3}\),
        \\injektiv fordi alle elementene i definisjonsmengden svarer til et unikt element i verdimengden, ikke surjektiv da alle elementene i \(\mathbb{R}\) ikke dekkes av formelen \({n\over 3}\).
        \\\(f^{-1}:\mathbb{R}\rightarrow\mathbb{Z}\).
        \item \(f:\mathbb{R}\rightarrow\mathbb{R}, f(x)={x\over3}\),
        \\bijektiv fordi alle elementene i definisjonsmengden svarer til et unikt element i verdimengden, og alle elementene i verdimengden svarer til et unikt element i definisjonsmengden.
        \\\(f^{-1}:\mathbb{R}\rightarrow\mathbb{R}, f(x) = 3x\).
        \item \(f:\mathbb{N}\rightarrow\mathbb{Z}, f(n)={-n\over2} \text{ hvis n er et partall, og } \frac{n+1}{2} \text{ hvis n er et oddetall.}\)
        \begin{table*}[h]
            \centering
            \begin{tabular}{cccccccccc}
                \(\mathbb{N:}\) & 0 & 1 & 2 & 3 & 4 & 5 & 6 & \(\cdots\) & \(\mathbb{N}\)\\
                & \(\downarrow\) & \(\downarrow\) & \(\downarrow\) & \(\downarrow\) & \(\downarrow\) & \(\downarrow\) & \(\downarrow\) & & \(\downarrow\)\\
                \(\mathbb{Z}:\) & 0 & 1 & -1 & 2 & -2 & 3 & -3 & \(\cdots\) & \(\mathbb{Z}\)\\
            \end{tabular}
            \\[1em]funksjonen \(f\) er bijektiv.
        \end{table*}        
    \end{enumerate}
\end{enumerate}

% \printbibliography[heading=bibintoc] % LAGER BIBLIOGRAFI
\end{document}
