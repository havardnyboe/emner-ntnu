% #region PREAMBEL OG PAKKER
\documentclass[a4paper, 12pt]{article}  % DOKUMENTKLASSE
\title{Øving 1}                         % TITTEL
\author{Håvard Solberg Nybøe}           % FORFATTER
\date{MA0301 -- \today}                 % DATO & FAG

\usepackage[english, norsk]{babel}      % NORSK SPRÅK
\usepackage[
    backend=biber,style=apa]{biblatex}  % BIBLIOGRAFI
\usepackage{csquotes}                   % PAKKE TIL BABEL
\addbibresource{bibliografi.bib}         % PATH TIL BIBLIOGRAFI
\usepackage[hidelinks]{hyperref}        % LENKER I TOC OG GENERELT
\usepackage[margin=1in]{geometry}       % VANLIG STØRRELSE MARGIN
\setlength{\parindent}{0em}             % SKILLER AVSNITT
\setlength{\parskip}{.8em}              % SKILLER AVSNITT
\usepackage{graphicx}                   % BILDER \includegraphics[OPTIONS]{PATH}
\usepackage{kantlipsum}                 % FYLLTEKST I KANT-STIL (kant[n-m])
\usepackage{amsfonts,                   % BLACKBOARD BOLD FONT (\mathbb{N})
amsmath,stmaryrd,amssymb}               % ANDRE MATTE PAKKER
\usepackage{circuitikz}                 % LOGISKE PORTER OG KRETSER & TikZ
\usepackage{import}                     % IMPORTER FILER (\import{PATH}{FILE})
\usepackage{caption}                    % PAKKE FOR BEDRE CAPTIONS I FIGURER
\usepackage{float}
% #endregion
\begin{document}

\maketitle

\emph{Ønsker retting}

\textbf{Exsercise 1}
\begin{table}[H]
    \begin{tabular}[H]{c|c|c|c|c|c|}
        (a) & $p$ & $q$ & $r$ & $q \lor r$ & $p \Rightarrow (q \lor r)$ \\
            & 0   & 0   & 0   & 0          & 1                          \\
            & 0   & 0   & 1   & 1          & 1                          \\
            & 0   & 1   & 0   & 1          & 1                          \\
            & 0   & 1   & 1   & 1          & 1                          \\
            & 1   & 0   & 0   & 0          & 0                          \\
            & 1   & 0   & 1   & 1          & 1                          \\
            & 1   & 1   & 0   & 1          & 1                          \\
            & 1   & 1   & 1   & 1          & 1                          \\
    \end{tabular}
\end{table}

\begin{table}[H]
    \begin{tabular}[H]{c|c|c|c|c|c|}
        (b) & $p$ & $q$ & $r$ & $p \Rightarrow q$ & $r \Rightarrow (p \Rightarrow q)$ \\
            & 0   & 0   & 0   & 1                 & 1                                 \\
            & 0   & 0   & 1   & 1                 & 1                                 \\
            & 0   & 1   & 0   & 1                 & 1                                 \\
            & 0   & 1   & 1   & 1                 & 1                                 \\
            & 1   & 0   & 0   & 0                 & 1                                 \\
            & 1   & 0   & 1   & 0                 & 0                                 \\
            & 1   & 1   & 0   & 1                 & 1                                 \\
            & 1   & 1   & 1   & 1                 & 1                                 \\
    \end{tabular}
\end{table}

\begin{table}[H]
    \begin{tabular}[H]{c|c|c|c|c|c|c|c|}
        (c) & $p$ & $q$ & $r$ & $p \oplus  r$ & $q \Rightarrow \neg r$ & $(q \Rightarrow \neg r) \lor (p \oplus r)$ & $p \Rightarrow (q \Rightarrow \neg r) \lor (p \oplus r)$ \\
            & 0   & 0   & 0   & 0             & 1                      & 1                                          & 1                                                        \\
            & 0   & 0   & 1   & 1             & 1                      & 1                                          & 1                                                        \\
            & 0   & 1   & 0   & 0             & 1                      & 1                                          & 1                                                        \\
            & 0   & 1   & 1   & 1             & 0                      & 1                                          & 1                                                        \\
            & 1   & 0   & 0   & 1             & 1                      & 1                                          & 1                                                        \\
            & 1   & 0   & 1   & 0             & 1                      & 1                                          & 1                                                        \\
            & 1   & 1   & 0   & 1             & 1                      & 1                                          & 1                                                        \\
            & 1   & 1   & 1   & 0             & 0                      & 0                                          & 0                                                        \\
    \end{tabular}
\end{table}

\newpage
\textbf{Exsercise 2}
\begin{enumerate}
    \item [(a)] $(p \lor q) \lor (p \Rightarrow q)$
          \\Hvis $p$ og/eller $q$ er sann, er $p \lor q$ sann. Hvis begge er usanne er $p \Rightarrow q$ sann.
          \\Utsagnet er en tautologi. $\blacksquare$

    \item [(b)] $\left( p \Rightarrow (q \land \neg q) \right) \land p$
          \\Anta at $p$ er sann. Skal utsagnet være sant, må $p \Rightarrow (q \land \neg q)$ være sant.
          \\ $(q \land \neg q) \equiv \mathrm{F}$, $p \Rightarrow \mathrm{F} \equiv \mathrm{F}$
          \\Utsagnet er en kontradiksjon. $\blacksquare$

    \item [(c)] $\left( p \Rightarrow (q \land \neg q) \right) \land p \Rightarrow r$
          \\Gitt resultatet i forrige oppgave er $\left( p \Rightarrow (q \land \neg q) \right) \land p$ usant.
          \\Utsagnet er en tautologi. $\blacksquare$
\end{enumerate}

\textbf{Exsercise 3}
\begin{enumerate}
    \item [(a)] $p \land q \Rightarrow r$
          \\$a$ is smaller than $b$ and $b$ is smaller than $c$ implies that $a$ is smaller than $c$.
          \\Logisk riktig antagelse.
    \item [(b)] $p \land q \Rightarrow u$
          \\$a$ is smaller than $b$ and $b$ is smaller than $c$ implies that $a$ is equal to $c$.
              \\Logisk feil antagelse da $a$ må være mindre enn $c$.
    \item [(c)] $(p \lor s) \land (q \lor t) \land u \Rightarrow s$
          \\$a$ smaller or equal to $b$ and $b$ smaller or equal to $c$ and $a$ equal to $c$ implies that $a$ is equal to $b$.
          \\Logisk riktig da alle vil ha samme verdi.
\end{enumerate}
% p: a is smaller than b.
% q: b is smaller than c.
% r: a is smaller than c.
% s: a is equal to b.
% t: b is equal to c.
% u: a is equal to c.

\textbf{Exsercise 4}
\\\newline Gitt at $q$ er sann (T), og at

$$ (q \Rightarrow ((p \lor \neg r) \land s)) \land (s \Rightarrow (r \land q)) $$

er en tautologi.
\\Tester med $r$ satt til sann.
\begin{align*}
    (q \Rightarrow ((p \lor \neg r) \land s))     & \land (s \Rightarrow (r \land q))                                                \\
    (q \Rightarrow ((p \lor \mathrm{F}) \land s)) & \land (s \Rightarrow \mathrm{T}), \quad s \equiv \mathrm{T}, p \equiv \mathrm{T} \\
    (q \Rightarrow (\mathrm{T} \land \mathrm{T})) & \land \mathrm{T}
\end{align*}
\begin{center}
    \text{$p$, $r$ og $s$ er sann.} $\blacksquare$
\end{center}

\newpage
\textbf{Exsercise 5}
\begin{enumerate}
    \item [(a)] Negate: $(p \land q) \Rightarrow (\neg r \lor \neg s)$
          \begin{flalign*}
              \neg((p \land q)      & \Rightarrow (\neg r \lor \neg s)) &  &                                                         \\
              \neg(\neg(p \land q)  & \lor (\neg r \lor \neg s))        &  & \textit{\small{(Material implication)}}         &  &  & \\
              \neg(\neg(p \land q)) & \land \neg(\neg r \lor \neg s)    &  & \textit{\small{(De Morgan)}}                            \\
              (p \land q)           & \land (r \land s)                 &  & \textit{\small{(De Morgan \& Double Negation)}}         \\
              p \land q             & \land r \land s                   &  & \textit{\small{(Associativity)}}                        \\
          \end{flalign*}
    \item [(b)] Negate: $p \Rightarrow (r \oplus s)$
          \begin{flalign*}
              \neg(p                        & \Rightarrow (r \oplus s))     &  &                                                         \\
              \neg(\neg p                   & \lor (r \oplus s))            &  & \textit{\small{(Material implication)}}         &  &  & \\
              p                             & \land \neg(r \oplus s)        &  & \textit{\small{(De Morgan \& Double Negation)}}         \\
              p \land \neg((r \lor s)       & \land (\neg r \lor \neg s))   &  & \textit{\small{(Replacement)}}                          \\
              p \land \neg(r \lor s)        & \lor \neg(\neg r \lor \neg s) &  & \textit{\small{(De Morgan)}}                            \\
              p \land (\neg r \land \neg s) & \lor ( r \land s)             &  & \textit{\small{(De Morgan)}}                            \\
          \end{flalign*}
\end{enumerate}

\textbf{Exsercise 6}
\begin{table}[H]
    \begin{tabular}{c|c|c|c|c|c|}
        AL1 (V.S.) & $\alpha$ & $\beta$ & $\gamma$ & $(\alpha \lor \beta)$ & $(\alpha \lor \beta) \lor \gamma $ \\
                   & 0        & 0       & 0        & 0                     & 0                                  \\
                   & 0        & 0       & 1        & 0                     & 1                                  \\
                   & 0        & 1       & 0        & 1                     & 1                                  \\
                   & 0        & 1       & 1        & 1                     & 1                                  \\
                   & 1        & 0       & 0        & 1                     & 1                                  \\
                   & 1        & 0       & 1        & 1                     & 1                                  \\
                   & 1        & 1       & 0        & 1                     & 1                                  \\
                   & 1        & 1       & 1        & 1                     & 1                                  \\
    \end{tabular}
\end{table}
\begin{table}[H]
    \begin{tabular}{c|c|c|c|c|c|}
        AL1 (H.S.) & $\alpha$ & $\beta$ & $\gamma$ & $(\beta \lor \gamma)$ & $\alpha \lor (\beta \lor \gamma) $ \\
                   & 0        & 0       & 0        & 0                     & 0                                  \\
                   & 0        & 0       & 1        & 1                     & 1                                  \\
                   & 0        & 1       & 0        & 1                     & 1                                  \\
                   & 0        & 1       & 1        & 1                     & 1                                  \\
                   & 1        & 0       & 0        & 0                     & 1                                  \\
                   & 1        & 0       & 1        & 1                     & 1                                  \\
                   & 1        & 1       & 0        & 1                     & 1                                  \\
                   & 1        & 1       & 1        & 1                     & 1                                  \\
    \end{tabular}
    \begin{equation*}
        (\alpha \lor \beta) \lor \gamma \equiv \alpha \lor (\beta \lor \gamma) \blacksquare
    \end{equation*}
\end{table}

\newpage
\begin{table}[H]
    \begin{tabular}{c|c|c|c|c|c|}
        AL2 (V.S.) & $\alpha$ & $\beta$ & $\gamma$ & $(\alpha \land \beta)$ & $(\alpha \land \beta) \land \gamma $ \\
                   & 0        & 0       & 0        & 0                      & 0                                    \\
                   & 0        & 0       & 1        & 0                      & 0                                    \\
                   & 0        & 1       & 0        & 0                      & 0                                    \\
                   & 0        & 1       & 1        & 0                      & 0                                    \\
                   & 1        & 0       & 0        & 0                      & 0                                    \\
                   & 1        & 0       & 1        & 0                      & 0                                    \\
                   & 1        & 1       & 0        & 1                      & 0                                    \\
                   & 1        & 1       & 1        & 1                      & 1                                    \\
    \end{tabular}
\end{table}

\begin{table}[H]
    \begin{tabular}{c|c|c|c|c|c|}
        AL2 (H.S.) & $\alpha$ & $\beta$ & $\gamma$ & $(\beta \land \gamma)$ & $\alpha \land (\beta \land \gamma) $ \\
                   & 0        & 0       & 0        & 0                      & 0                                    \\
                   & 0        & 0       & 1        & 0                      & 0                                    \\
                   & 0        & 1       & 0        & 0                      & 0                                    \\
                   & 0        & 1       & 1        & 1                      & 0                                    \\
                   & 1        & 0       & 0        & 0                      & 0                                    \\
                   & 1        & 0       & 1        & 0                      & 0                                    \\
                   & 1        & 1       & 0        & 0                      & 0                                    \\
                   & 1        & 1       & 1        & 1                      & 1                                    \\
    \end{tabular}
    \begin{equation*}
        (\alpha \land \beta) \land \gamma \equiv \alpha \land (\beta \land \gamma) \blacksquare
    \end{equation*}
\end{table}


\textbf{Exsercise 7}

\begin{table}[H]
    \begin{tabular}{c|c|c|c|c|}
        (a) & $p$ & $q$ & $p \lor q$ & $p \Rightarrow (p \lor q)$ \\
            & 0   & 0   & 0          & 1                          \\
            & 0   & 1   & 1          & 1                          \\
            & 1   & 0   & 1          & 1                          \\
            & 1   & 1   & 1          & 1                          \\
    \end{tabular}
    \begin{center}
        Utsagnet $p \Rightarrow (p \lor q)$ er en tautologi. $\blacksquare$
    \end{center}
\end{table}

\begin{table}[H]
    \begin{tabular}{c|c|c|c|c|}
        (b) & $p$ & $q$ & $p \lor q$ & $\neg(p \Rightarrow (p \lor q))$ \\
            & 0   & 0   & 0          & 0                                \\
            & 0   & 1   & 1          & 0                                \\
            & 1   & 0   & 1          & 1                                \\
            & 1   & 1   & 1          & 0                                \\
    \end{tabular}
    \begin{center}
        Utsagnet $p \Rightarrow (p \lor q)$ er tilfredstillbar. $\blacksquare$
    \end{center}
\end{table}

\begin{table}[H]
    \begin{tabular}{c|c|c|c|c|}
        (c) & $p$ & $q$ & $p \Rightarrow q$ & $p \Rightarrow (p \Rightarrow q)$ \\
            & 0   & 0   & 1                 & 1                                 \\
            & 0   & 1   & 1                 & 1                                 \\
            & 1   & 0   & 0                 & 0                                 \\
            & 1   & 1   & 1                 & 1                                 \\
    \end{tabular}
    \begin{center}
        Utsagnet $p \Rightarrow (p \Rightarrow q)$ er tilfredstillbar. $\blacksquare$
    \end{center}
\end{table}

\textbf{Exsercise 8}

\begin{enumerate}
    \item \(\neg p \Rightarrow (q \Leftrightarrow r )\)
    \item \(r \Rightarrow \neg p\)
    \item \(\neg r \oplus (p \land q)\)
    \item \(p \Rightarrow (r \land q)\)
    \item \(\neg q \oplus r\)
\end{enumerate}

\end{document}
