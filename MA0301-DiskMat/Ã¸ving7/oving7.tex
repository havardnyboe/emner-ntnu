% #region PREAMBEL OG PAKKER
\documentclass[a4paper, 12pt]{article}  % DOKUMENTKLASSE
\title{Øving 7}                         % TITTEL
\author{Håvard Solberg Nybøe}           % FORFATTER
\date{MA0301 -- \today}                 % DATO & FAG

\usepackage[english, norsk]{babel}      % NORSK SPRÅK
\usepackage[
    backend=biber,style=apa]{biblatex}  % BIBLIOGRAFI
\usepackage{csquotes}                   % PAKKE TIL BABEL
\addbibresource{bibliografi.bib}         % PATH TIL BIBLIOGRAFI
\usepackage[hidelinks]{hyperref}        % LENKER I TOC OG GENERELT
\usepackage[margin=1in]{geometry}       % VANLIG STØRRELSE MARGIN
\setlength{\parindent}{0em}             % SKILLER AVSNITT
\setlength{\parskip}{.8em}              % SKILLER AVSNITT
\usepackage{graphicx}                   % BILDER \includegraphics[OPTIONS]{PATH}
\usepackage{kantlipsum}                 % FYLLTEKST I KANT-STIL (kant[n-m])
\usepackage{amsfonts,                   % BLACKBOARD BOLD FONT (\mathbb{N})
amsmath,stmaryrd,amssymb}               % ANDRE MATTE PAKKER
\usepackage{circuitikz}                 % LOGISKE PORTER OG KRETSER & TikZ
\usepackage{import}                     % IMPORTER FILER (\import{PATH}{FILE})
\usepackage{caption}                    % PAKKE FOR BEDRE CAPTIONS I FIGURER
\usepackage{float}                       % FLYTT FIGURER 
% #endregion
\begin{document}

\maketitle

\begin{enumerate}
    \item [\boxed{1}]
        Tallene \(a_0, a_1\) og \(a_2\) er alle multiplum av 3, \(0(3), 2(3), 3(3)\), altså er de delelig på 3.
        Formelen \(a_n = a_{n-1} + a_{n-3}\) legger kun sammen leddene ergo vil summen alltid være et multiplum av 3 og dermed delelig på 3. \(\square\)
    \item [\boxed{2}] 
    \begin{flalign*}
        \text{Grunnsteg} &: \quad p_0: a = 1, b = 1, \quad 3 \cdot 1 + 5 \cdot 1 = 8 &\\
        &\textcolor{white}{:::} \quad p_1: a = 3, b = 0, \quad 3 \cdot 3 + 5 \cdot 0 = 9 \\
        &\textcolor{white}{:::} \quad p_2: a = 0, b = 2, \quad 3 \cdot 0 + 5 \cdot 2 = 10 \\
        &\textcolor{white}{:::} \quad p_3: a = 2, b = 1, \quad 3 \cdot 2 + 5 \cdot 1 = 11 \\
        &\textcolor{white}{:::} \quad p_4: a = 4, b = 0, \quad 3 \cdot 4 + 5 \cdot 0 = 12 \\
        &\textcolor{white}{:::} \quad p_5: a = 1, b = 2, \quad 3 \cdot 1 + 5 \cdot 2 = 13 \\
        &\textcolor{white}{:::} \quad p_6: a = 3, b = 1, \quad 3 \cdot 3 + 5 \cdot 1 = 14 \\
        &\textcolor{white}{:::} \quad p_7: a = 0, b = 3, \quad 3 \cdot 0 + 5 \cdot 3 = 15 \\
        \text{Bevis} &: \quad \text{Anta } k = 3a + 5b, \\
        &\textcolor{white}{:::} \quad p_7: a = 0, b = 3, \quad 3 \cdot 0 + 5 \cdot 3 = 15 \\
        &\textcolor{white}{:::} \quad k = 3 \cdot (a_{p_k\text{ mod }8}+\frac{k - (k\text{ mod }8)}{8} - 1) +  5 \cdot (b_{p_k\text{ mod }8}+\frac{k - (k\text{ mod }8)}{8} - 1) \\
        \\
        \text{ex.} &: \quad 33 = 3 \cdot (3 + \frac{33 - 1}{8} - 1) + 5 \cdot (0 + \frac{33 - 1}{8} - 1) \\
        &\textcolor{white}{:::} \quad 33 = 3 \cdot (3 + 4 - 1) + 5 \cdot (0 + 4 - 1) \\
        &\textcolor{white}{:::} \quad 33 = 18 + 15 \\
        &\textcolor{white}{:::} \quad 33 = 33 \\
    \end{flalign*}
    \item [\boxed{4}] Primtall t.o.m. \(11\): \(\{2, 3, 5, 7, 11\}\)
    \begin{flalign*} 
        \text{Grunnsteg} &: \quad 4 = 2 \cdot 2 &\\
        &\textcolor{white}{:::} \quad 6 = 2 \cdot 3 \\
        &\textcolor{white}{:::} \quad 8 = 2 \cdot 4 \\
        &\textcolor{white}{:::} \quad 9 = 3 \cdot 3 \\
        &\textcolor{white}{:::} \quad 10 = 2 \cdot 5 \\
        \text{Bevis} &: \quad \text{Anta at det finnes minst ett heltall } a > 1\\ 
        &\textcolor{white}{:::} \quad \text{som ikke kan skrives som et produkt av primtall} \\
        &\textcolor{white}{:::} \quad \text{hvis et tall } b \text{ kan skrives som } 1 < b < a\\
        &\textcolor{white}{:::} \quad \text{så kan } b \text{ skrives som et produkt av primtall } p_1 \cdot p_2 \cdots p_n\\
        &\textcolor{white}{:::} \quad \text{da er a åpenbart ikke et primtall} \\
        &\textcolor{white}{:::} \quad \text{siden hvis det var det kunne det skrives som } a = a \quad\square\\
    \end{flalign*}
    \item [\boxed{7}] \(b \rightarrow c \rightarrow c \rightarrow d\) siden c er refleksiv.
    \item [\boxed{8}]
    \begin{enumerate}
        \item a: inn = \(b \rightarrow a\), ut = \(a \rightarrow c\)
        \\b: inn = \(d \rightarrow b\), ut = \(b \rightarrow a, b \rightarrow c\)
        \\c: inn = \(a \rightarrow c\), ut = \(c \rightarrow d, c \rightarrow e\)
        \\d: inn = \(c \rightarrow d, a \rightarrow d\), ut = \(d \rightarrow b, d \rightarrow e\)
        \\e: inn = \(c \rightarrow e, d \rightarrow e\)
        \item \(a \rightarrow b \rightarrow c \rightarrow d \rightarrow a, 
        \\b \rightarrow c \rightarrow d \rightarrow b\)
        \\ en sykel passerer gjennom \(a\), to sykler passerer gjennom \(b, c, d\), ingen sykler passerer gjennom \(e\)
        \item 
        \item 4, \(a \rightarrow c \rightarrow d \rightarrow b \rightarrow a,
        \\a \rightarrow d \rightarrow b \rightarrow c \rightarrow e\)
    \end{enumerate}
\end{enumerate}

% \printbibliography[heading=bibintoc] % LAGER BIBLIOGRAFI
\end{document}
