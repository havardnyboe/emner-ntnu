% #region PREAMBEL OG PAKKER
\documentclass[a4paper, 12pt]{article}  % DOKUMENTKLASSE
\title{FI1005 Øving 2}                         % TITTEL
\author{Håvard Solberg Nybøe}           % FORFATTER
\date{\today}                           % DATO & FAG

% Minimal preamble med nødvendige pakker
\usepackage[english, norsk]{babel}      % NORSK SPRÅK
\usepackage[
    backend=biber,style=ieee]{biblatex} % BIBLIOGRAFI
\usepackage{csquotes}                   % PAKKE TIL BABEL
\addbibresource{bibliografi.bib}        % PATH TIL BIBLIOGRAFI
\usepackage[hidelinks]{hyperref}        % LENKER I TOC OG GENERELT
\usepackage[margin=1in]{geometry}       % VANLIG STØRRELSE MARGIN
\setlength{\parindent}{0em}             % SKILLER AVSNITT
\setlength{\parskip}{.8em}              % SKILLER AVSNITT
\usepackage{graphicx}                   % BILDER \includegraphics[OPTIONS]{PATH}
\usepackage{kantlipsum}                 % FYLLTEKST I KANT-STIL (kant[n-m])
\usepackage{amsfonts,                   % BLACKBOARD BOLD FONT (\mathbb{N})
amsmath,stmaryrd,amssymb}               % ANDRE MATTE PAKKER
\usepackage{caption}                    % PAKKE FOR BEDRE CAPTIONS I FIGURER
\usepackage{float}                      % FLYTT FIGURER 
\usepackage[UTF8]{ctex}                 % UTF8 TEGN
\usepackage{minted}                     % KODEBLOKKER OG SLIKT
% #endregion

\begin{document}

\maketitle
% \tableofcontents % INNHOLDSFORTEGNELSE


\begin{enumerate}
    \item [\boxed{1}]
    \begin{enumerate}
        \item Det snør, og det blåser.
        \\ \emph{Setningen er kontingent.}
        \item Enten blåser det, eller så blåser det ikke.
        \\ \emph{Setningen er en tautologi.}
        \item \(A \to (A \land \neg B)\)
        \\ \emph{Setningen er kontingent.}
        \item \((A \land B) \land (\neg A \lor \neg B)\)
        \\ \emph{Setningen er en kontradiksjon.}
        \item Det er glatt, men hvis vi strør fortauet, er det ikke glatt.
        \\ \emph{Setningen er kontingent.}
    \end{enumerate}
    \item [\boxed{2}]
    \begin{enumerate}
        \item 
        \begin{tabular}[t]{c|c}
            \(A\) & \(A \land \neg A \) \\
            \hline
            0 & 0 \\
            1 & 0 \\
        \end{tabular}
        \hspace*{1em}
        \begin{tabular}[t]{c|c}
            \(A\) & \(A \to A \) \\
            \hline
            0 & 1 \\
            1 & 1 \\
        \end{tabular}
        \\[1em]
        \((A \land \neg A)\) og \(A \to A\) er ikke logisk ekvivalente.
        \item
        \begin{tabular}[t]{cc|c}
            \(A\) & \(B\) & \(A \to B \) \\
            \hline
            0 & 0 & 1 \\
            0 & 1 & 1 \\
            1 & 0 & 0 \\
            1 & 1 & 1 \\
        \end{tabular}
        \hspace*{1em}
        \begin{tabular}[t]{cc|c}
            \(A\) & \(B\) & \(\neg A \lor B\) \\
            \hline
            0 & 0 & 1 \\
            0 & 1 & 1 \\
            1 & 0 & 0 \\
            1 & 1 & 1 \\
        \end{tabular}
        \\[1em]
        \(A \to B\) og \(\neg A \lor B\) er logisk ekvivalente.
        \item
        \begin{tabular}[t]{ccc|c}
            \(A\) & \(B\) & \(C\) & \(A \land (B \lor C) \) \\
            \hline
            0 & 0 & 0 & 0 \\
            0 & 0 & 1 & 0 \\
            0 & 1 & 0 & 0 \\
            0 & 1 & 1 & 0 \\
            1 & 0 & 0 & 0 \\
            1 & 0 & 1 & 1 \\
            1 & 1 & 0 & 1 \\
            1 & 1 & 1 & 1 \\
        \end{tabular}
        \hspace*{1em}
        \begin{tabular}[t]{ccc|c}
            \(A\) & \(B\) & \(C\) & \(\neg A \lor (\neg B \to C) \) \\
            \hline
            0 & 0 & 0 & 1 \\
            0 & 0 & 1 & 1 \\
            0 & 1 & 0 & 1 \\
            0 & 1 & 1 & 1 \\
            1 & 0 & 0 & 0 \\
            1 & 0 & 1 & 1 \\
            1 & 1 & 0 & 1 \\
            1 & 1 & 1 & 1 \\
        \end{tabular}
        \\[1em]
        \(A \land (B \lor C)\) og \(\neg A \lor (\neg B \to C)\) er ikke logisk ekvivalente.
        \item
        \begin{tabular}[t]{ccc|c}
            \(A\) & \(B\) & \(C\) & \(A \to (B \lor C) \) \\
            \hline
            0 & 0 & 0 & 1 \\
            0 & 0 & 1 & 1 \\
            0 & 1 & 0 & 1 \\
            0 & 1 & 1 & 1 \\
            1 & 0 & 0 & 0 \\
            1 & 0 & 1 & 1 \\
            1 & 1 & 0 & 1 \\
            1 & 1 & 1 & 1 \\
        \end{tabular}
        \hspace*{1em}
        \begin{tabular}[t]{ccc|c}
            \(A\) & \(B\) & \(C\) & \(\neg A \lor (\neg B \to C) \) \\
            \hline
            0 & 0 & 0 & 1 \\
            0 & 0 & 1 & 1 \\
            0 & 1 & 0 & 1 \\
            0 & 1 & 1 & 1 \\
            1 & 0 & 0 & 0 \\
            1 & 0 & 1 & 1 \\
            1 & 1 & 0 & 1 \\
            1 & 1 & 1 & 1 \\
        \end{tabular}
        \\[1em]
        \(A \to (B \lor C)\) og \(\neg A \lor (\neg B \to C)\) er logisk ekvivalente.
    \end{enumerate}
    \item [\boxed{3}]
    \begin{enumerate}
        \item
        \begin{tabular}[t]{cc|cc|c}
            \(A\) & \(B\) & \(A\) & \(A \to B\) & \(B\)\\
            \hline
            0 & 0 & 0 & 1 & 0 \\
            0 & 1 & 0 & 1 & 1 \\
            1 & 0 & 1 & 0 & 0 \\
            1 & 1 & 1 & 1 & 1 \\
        \end{tabular}
    \end{enumerate}
    \item [\boxed{4}]
    \begin{enumerate}
        \item
        \begin{tabular}[t]{cccc|c}
            \(A\) & \(B\) & \(A \lor B\) & \(\neg B\) & \(A \to B\) \\
            \hline
            0 & 0 & 0 & 1 & 1 \\
            0 & 1 & 1 & 0 & 1 \\
            1 & 0 & 1 & 1 & 0 \\
            1 & 1 & 1 & 0 & 1 \\            
        \end{tabular}
    \end{enumerate}
\end{enumerate}

% \printbibliography[heading=bibintoc] % LAGER BIBLIOGRAFI
\end{document}