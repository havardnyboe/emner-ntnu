% #region PREAMBEL OG PAKKER
\documentclass[a4paper, 12pt]{article}  % DOKUMENTKLASSE
\title{FI1005 Øving 8}                         % TITTEL
\author{Håvard Solberg Nybøe}           % FORFATTER
\date{\today}                           % DATO & FAG

% Minimal preamble med nødvendige pakker
\usepackage[english, norsk]{babel}      % NORSK SPRÅK
\usepackage[
    backend=biber,style=ieee]{biblatex} % BIBLIOGRAFI
\usepackage{csquotes}                   % PAKKE TIL BABEL
\addbibresource{bibliografi.bib}        % PATH TIL BIBLIOGRAFI
\usepackage[hidelinks]{hyperref}        % LENKER I TOC OG GENERELT
\usepackage[margin=1in]{geometry}       % VANLIG STØRRELSE MARGIN
\setlength{\parindent}{0em}             % SKILLER AVSNITT
\setlength{\parskip}{.8em}              % SKILLER AVSNITT
\usepackage{graphicx}                   % BILDER \includegraphics[OPTIONS]{PATH}
\usepackage{kantlipsum}                 % FYLLTEKST I KANT-STIL (kant[n-m])
\usepackage{amsfonts,                   % BLACKBOARD BOLD FONT (\mathbb{N})
amsmath,stmaryrd,amssymb}               % ANDRE MATTE PAKKER
\usepackage{caption}                    % PAKKE FOR BEDRE CAPTIONS I FIGURER
\usepackage{float}                      % FLYTT FIGURER 
\usepackage[UTF8]{ctex}                 % UTF8 TEGN
\usepackage{minted}                     % KODEBLOKKER OG SLIKT
% #endregion

\begin{document}

\maketitle
% \tableofcontents % INNHOLDSFORTEGNELSE


\begin{enumerate}
    \item [\boxed{1}]
    \begin{enumerate}
        \item \(x\) er minst like gammel som \(y\) (Refleksiv med ikke symmetrisk)
        \item \(x\) er venn med \(y\) (Symmetrisk men ikke transitiv)
        \item \(x\) og \(y\) er hel-søsken (Transitiv og symmetrisk, men ikke refleksiv)
    \end{enumerate}
    \item [\boxed{2}]
    \begin{enumerate}
        \item [(b)] 
    \end{enumerate}
    \item [\boxed{3}]
    \begin{enumerate}
        \item \(\exists x [x = x]\) \\
        \(a = a\) \\
        \(\exists x [x = x]\)
    \end{enumerate}
    \item [\boxed{4}]
    \begin{enumerate}
        \item 
    \end{enumerate}
\end{enumerate}

% \printbibliography[heading=bibintoc] % LAGER BIBLIOGRAFI
\end{document}