\documentclass[a4paper, 12pt]{article}  % DOKUMENTKLASSE
\title{Dokument}                        % TITTEL
\author{EXPH0300}                       % FORFATTER
\date{\today}                           % DATO & FAG

\usepackage[english, norsk]{babel}      % NORSK SPRÅK
\usepackage[backend=biber,style=apa]{biblatex}  % BIBLIOGRAFI
\usepackage{csquotes}                   % PAKKE TIL BABEL
\addbibresource{references.bib}         % PATH TIL BIBLIOGRAFI
\usepackage[hidelinks]{hyperref}        % LENKER I TOC OG GENERELT
\usepackage[margin=1in]{geometry}       % VANLIG STØRRELSE MARGIN
\setlength{\parindent}{0em}             % SKILLER AVSNITT
\setlength{\parskip}{.8em}              % SKILLER AVSNITT
\usepackage{graphicx}                   % BILDER \includegraphics[OPTIONS]{PATH}
\usepackage{kantlipsum}                 % FYLLTEKST I KANT-STIL (kant[n-m])
\usepackage{amsfonts,                   % BLACKBOARD BOLD FONT (\mathbb{N})
amsmath,stmaryrd,amssymb}               % ANDRE MATTE PAKKER
\usepackage{import}                     % IMPORTER FILER (\import{PATH}{FILE})
\usepackage{caption}                    % PAKKE FOR BEDRE CAPTIONS I FIGURER
\renewcommand{\baselinestretch}{1.5}

\begin{document}

\maketitle
\tableofcontents % INNHOLDSFORTEGNELSE
\newpage

\section*{Disposisjon}
\begin{itemize}
    \item \textbf{Bevisthetsfilosofi}
          \begin{itemize}
              \item Omhandler mentale hendelser og egenskaper
              \item Sentrale retninger:
                    \begin{itemize}
                        \item Dualisme (Descartes)
                              \begin{itemize}
                                  \item Descartes' argument for dualisme
                                  \item Kunnskapsargumentet
                              \end{itemize}
                        \item Fysikalisme
                        \item Dobbeltaspektteori
                        \item Funksjonalisme (ikke i pensum)
                    \end{itemize}
              \item Relevante filosofer:
                    \begin{itemize}
                        \item René Descartes (Cogito ergo sum)
                        \item Immanuel Kant?
                    \end{itemize}
          \end{itemize}
    \item \textbf{Kan maskiner tenke?}
          \begin{itemize}
              \item Viktig spørsmål innen bevisthetsfilosofi og IT
              \item Diskusjonen har dreid seg om hva det innebærer å kunne tenke
                    \begin{itemize}
                        \item Turing-testen besvarer dette
                    \end{itemize}
              \item John Searles tankeekspeiment <<det kinesiske rommet>> argumenterer \\mot Turing
              \item Relevante filosofer/personer:
                    \begin{itemize}
                        \item Alan Turing
                        \item John Searle
                    \end{itemize}
          \end{itemize}
\end{itemize}

\section*{Bevisthetsfilosofi}
\addcontentsline{toc}{section}{Bevisthetsfilosofi}
\lipsum[1-2]

\section*{Kan maskiner tenke?}
\addcontentsline{toc}{section}{Kan maskiner tenke?}
\lipsum[1-2]

\parencite{neufeld2013reviews}

\printbibliography[heading=bibintoc] % LAGER BIBLIOGRAFI
\end{document}
