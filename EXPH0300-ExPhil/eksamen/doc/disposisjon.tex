\section*{Disposisjon}
\addcontentsline{toc}{section}{Disposisjon}
\begin{itemize}
    \item \textbf{Bevisthetsfilosofi}
          \begin{itemize}
              \item Omhandler mentale hendelser og egenskaper
              \item Sentrale begreper:
                    \begin{itemize}
                        \item Dualisme (Descartes)
                              \begin{itemize}
                                  \item Det finnes to forskjellige typer ting: fysiske ting og mentale ting
                                  \item Descartes' argument for dualisme
                                  \item Kunnskapsargumentet
                                  \item Monisme - finnes bare en ting \(\rightarrow\) Fysikalisme, Idealisme og Nøytral monisme
                              \end{itemize}
                        \item Fysikalisme
                              \begin{itemize}
                                  \item
                                  \item Funksjonalisme (ikke i pensum)
                              \end{itemize}
                        \item Dobbeltaspektteori
                    \end{itemize}
              \item Relevante filosofer:
                    \begin{itemize}
                        \item René Descartes (Cogito ergo sum)
                        \item Immanuel Kant?
                    \end{itemize}
          \end{itemize}
    \item \textbf{Kan maskiner tenke?}
          \begin{itemize}
              \item Viktig spørsmål innen bevisthetsfilosofi og IT
              \item Diskusjonen har dreid seg om hva det innebærer å kunne tenke
                    \begin{itemize}
                        \item Turing-testen besvarer dette
                    \end{itemize}
              \item John Searles tankeekspeiment <<det kinesiske rommet>> argumenterer \\mot Turing
              \item Relevante filosofer/personer:
                    \begin{itemize}
                        \item Alan Turing
                        \item John Searle
                    \end{itemize}
          \end{itemize}
\end{itemize}