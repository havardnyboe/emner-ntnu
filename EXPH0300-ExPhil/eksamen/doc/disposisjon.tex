\section*{Disposisjon}
\addcontentsline{toc}{section}{Disposisjon}
\subsection*{Bevisthetsfilosofi \textnormal{\textit{(Forelesning 10, Ons. 03.11)}}}
\addcontentsline{toc}{subsection}{Bevisthetsfilosofi}
\begin{itemize}
    \item Omhandler mentale hendelser og egenskaper
    \item Sentrale begreper:
          \begin{itemize}
              \item Dualisme \textit{(Descartes)}
                    \begin{itemize}
                        \item Det finnes to forskjellige typer ting: fysiske ting og mentale ting
                        \item Descartes' argument for dualisme
                        \item Kunnskapsargumentet
                        \item Monisme - finnes bare en ting \(\rightarrow\) Fysikalisme, Idealisme og Nøytral monisme
                    \end{itemize}
              \item Fysikalisme
                    \begin{itemize}
                        \item Identitetsteori - Mentale tilstander er identiske med fysiske tilstander
                              \begin{itemize}
                                  \item Reduktiv
                              \end{itemize}
                        \item Funksjonalisme - Mentale tilstander er en bestemt funksjon eller rolle i et fysisk system
                              \begin{itemize}
                                  \item Ikke-reduktiv
                              \end{itemize}
                    \end{itemize}
              \item Dobbeltaspektteori
          \end{itemize}
    \item Relevante filosofer:
          \begin{itemize}
              \item René Descartes \textit{(Cogito ergo sum)}
              \item Immanuel Kant?
          \end{itemize}
\end{itemize}
\subsection*{Kan maskiner tenke? \textnormal{\textit{(Forelesning 11, Ons. 10.11)}}}
\addcontentsline{toc}{subsection}{Kan maskiner tenke?}
\begin{itemize}
    \item Viktig spørsmål innen bevisthetsfilosofi og IT
    \item Diskusjonen har dreid seg om hva det innebærer å kunne tenke
          \begin{itemize}
              \item Turing-testen besvarer dette
          \end{itemize}
    \item John Searles tankeeksperiment <<det kinesiske rommet>> argumenterer \\mot Turing
    \item Relevante filosofer/personer:
          \begin{itemize}
              \item Alan Turing
              \item John Searle
          \end{itemize}
\end{itemize}

\subsection*{Innledning}
\begin{itemize}
    \item Hva skal besvarelsen ta for seg? 
\end{itemize}

\subsection*{Hoveddel}
\begin{itemize}
    \item Framstille pensum/fagstoff, 
    \item diskutere/drøfte problemstilling
\end{itemize}


\subsection*{Konklusjon}
\begin{itemize}
    \item Hva kom jeg fram til i diskusjonen/argumentasjonen?
\end{itemize}
