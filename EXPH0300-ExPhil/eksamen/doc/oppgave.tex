\section{Oppgave}
\subsection{Bakgrunn}\label{bakgrunn}

Bevissthetsfilosofi er den delen av filosofien som handler om mentale
hendelser og egenskaper, og deres relasjon til kroppen. Sentrale
bevissthetsfilosofiske retninger som behandles i pensum er fysikalisme,
dualisme og dobbeltaspektteori. En fjerde gruppe av
bevissthetsfilosofiske teorier, som ikke behandles i pensum, er
funksjonalisme.

Kan maskiner tenke? Dette spørsmålet har blitt diskutert innenfor
bevissthetsfilosofi og informasjonsteknologi de seneste årene.
Diskusjonen har dels dreid seg om hva det innebærer å kunne tenke. Den
såkalte Turing-testen gir et mulig svar på dette spørsmålet. John Searle
bruker et tankeeksperiment som gjerne kalles ``det kinesiske rommet''
for å argumentere mot at Turing-testen gir et treffende mål på hva det
vil si å kunne tenke.

\subsection{Oppgave}\label{oppgave}

\begin{enumerate}
  \def\labelenumi{\alph{enumi}.}
  \item
        I pensumteksten sier Thomas Nagel at ``argumentene mot en ren fysisk
        teori om bevissthet er sterke nok til å sannsynliggjøre at en fysisk
        teori om hele virkeligheten er umulig''. Bruk pensum til å diskutere
        denne påstanden.
  \item
        Diskuter om det er prinsipielt mulig at maskiner kan tenke og
        eventuelt om de kan være bevisste.
\end{enumerate}

Både a og b skal besvares, og svarene bør settes i sammenheng med
hverandre. Gjør grundig rede for de filosofiske teoriene og begrepene du
velger å bruke.

\subsection{Relevant pensumlitteratur}\label{relevant-pensumlitteratur}

\begin{itemize}
  \item
        Nagel, T. (2003). Problemet med forholdet mellom kropp og sinn. I T.
        Nagel, Hva er meningen? En kort innføring i filosofi (s. 31-39). Oslo:
        Libro.
  \item
        Kiran, A. H. (2021). Kan maskiner tenke? NTNU
\end{itemize}

\subsection{Anbefalt
  tilleggslitteratur}\label{anbefalt-tilleggslitteratur}

\begin{itemize}
  \item
        Internet Encyclopedia Internet Encyclopedia of philosophy ---
        Artificial Intelligence \url{https://iep.utm.edu/art-inte/}
\end{itemize}

\subsection{Formelle krav}\label{formelle-krav}

\begin{itemize}
  \item
        Besvarelsen blir i hovedsak bedømt ut fra hvor godt den demonstrerer
        forståelse av og ferdigheter i å bruke relevant pensumstoff.
  \item
        Besvarelsene må følge god kildehenvisningspraksis
  \item
        Oppgaven skal besvares med basis i pensum. De filosofiske kildene som
        brukes må tilfredsstille akademiske kvalitetskrav. Eksempler på slike
        kilder er pensum, lærebøker på universitetsnivå, fagfellevurderte
        artikler og oppslagsverk, slik som Stanford Encyclopedia of Philosophy
        og Internet Encyclopedia of Philosophy. OBS! Wikipedia, filosofi.no,
        repetisjonshefter og forelesningsnotater er eksempler på kilder som
        det ikke er akseptabelt å referere til. Store norske leksikon (SNL) er
        en akseptabel tilleggskilde, men kun for temaer som ikke er dekket i
        pensum.
\end{itemize}
