\section{Innledning}

Denne besvarelsen skal ta for seg de to deloppgavene som er beskrevet i oppgaveteksten. 
Den første delen av besvarelsen handler i all hovedsak om Thomas Nagles argumenter mot en ren fysisk teori om bevissthet, hvorfor disse argumentene muligens ikke holder helt til.
Den andre delen av besvarelsen skal ta for seg spørsmålet <<\textit{Kan maskiner tenke?}>> ved å først prøve å forstå hva det vil si å tenke, og deretter å forstå hva som skiller et tenkende menneske og en tenkende maskin. 

Besvarelsen kommer til svare på oppgaven ved å knytte problemstillingene opp mot relevant teori fra pensum og fra andre relevante tekster. Mer spesifikt blir Nagels argumenter og tankeeksperiment knyttet opp mot argumenter for identitetsteori og funksjonalisme, og blir drøftet ut i fra det. Spørsmålet om maskiner kan tenke blir brutt ned i to deler; hva det vil si og tenke, og om bevissthet er en vesentlig del av det å tenke. Det blir knyttet opp til René Descartes argumenter om tvil og tanke, Alan Turings imitasjonslek, og John Searles begrep <<\textit{weak AI}>>.