\section{Bevissthetsfilosofi}

\subsection{Hva er bevissthet?}

Bevissthet menes å være i følge filosofer som David Chalmers et av de, hvis ikke det, 
vanskeligste problemet vitenskapen har å bryne seg på \autocite{Chalmers1995}. 
En måte å definere hva bevissthet er kan være å se på forholdet mellom bevisstheten og hjernen.
Bevisstheten er avhengig av inntrykk den får fra hjernen enten direkte eller i form av sanser fra omverdenen. 
Bevisstheten er derfor opplevelsen man får av å sanse, tenke eller føle. 

\subsection{Nagels dualistiske argument}

Det er blant annen mange av disse argumentene Thomas Nagel bruker i sin tekst, 
der han forsøker å se på problemet med forholdet mellom kropp og sinn.
I teksten bruker Nagel et tenkt eksempel med en sjokolade,
der man tar en bit av sjokoladen og kjenner smaken av denne sjokoladen. 
Nagel argumenterer med at følelsen av denne smaken er adskilt fra de fysiske prosessene i hjerne. 
Og selv i det helt usannsynlige tilfellet der en person åpnet skallen til den som spiser sjokoladen og smaker på hjernemassen, 
kunne man umulig smake det samme som den som spiste sjokoladen fordi opplevelsen den personen med sjokoladen hadde er unik fordi den ikke bare kan komme av tilstander fra hjernen, 
men heller noe utenfor kroppen din \autocite[32-34]{Nagel2003}. 

Nagels ideer om bevissthet minner en del om Descartes' argumenter for dualisme.
\textit{Dualisme} handler nettopp om det Nagel framlegger; 
skillet mellom det fysiske, det som eksisterer i vår verden, og det mentale, det som eksisterer i vår bevissthet. 
Eller som Descartes kaller det, det \textit{objektive} og det \textit{subjektive} \autocite[156]{Dybvig2003}.


\subsection{Argumenter for en fysisk teori om bevissthet}


Nagel forklarer først at detaljene for fysikalistisk teori om bevissthet er ``\emph{opp til vitenskapen å avdekke}'' som i seg selv veldig riktig, 
da dette ikke lengre er et filosofisk spørsmål men heller et vitenskapelig.
,
senere rent benekter han at fysiske hendelser i hjernen kan utgjøre en smaksopplevelse.
\begin{pquotation}{\cite[36]{Nagel2003}}
    ``Det er ikke mulig at et stort antall fysiske hendelser i hjernen,
uansett hvor kompliserte de måtte være, 
kan være de komponentene som utgjør smaksopplevelsene våre.''
\end{pquotation}

Hei dette er en test \autocite{snl:Böhmen} for å se om det funker

Nevral identitetsteori - en mental tilstand er ikke noe mer enn en nevral tilstand.
På lik linje med at en tilstand i en datamaskin ikke noe mer enn en binær tallstreng.