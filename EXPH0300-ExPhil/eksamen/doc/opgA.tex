\section{Bevissthetsfilosofi}

\subsection{Hva er bevissthet?}

Bevissthet menes å være i følge filosofer som David Chalmers et av de, hvis ikke det, 
vanskeligste problemet vitenskapen har å bryne seg på \autocite{Chalmers1995}. 
En måte å definere hva bevissthet er kan være å se på forholdet mellom bevisstheten og hjernen.
Bevisstheten er avhengig av inntrykk den får fra hjernen enten direkte eller i form av sanser fra omverdenen. 
Bevisstheten er derfor opplevelsen man får av å sanse, tenke eller føle. 

\subsection{Nagels dualistiske argument}

Det er blant annen mange av disse argumentene Thomas Nagel bruker i sin tekst, 
der han forsøker å se på problemet med forholdet mellom kropp og sinn.
I teksten bruker Nagel et tenkt eksempel med en sjokolade,
der man tar en bit av sjokoladen og kjenner smaken av denne sjokoladen. 
Nagel argumenterer med at følelsen av denne smaken er adskilt fra de fysiske prosessene i hjerne. 
Og selv i det helt usannsynlige tilfellet der en person åpnet skallen til den som spiser sjokoladen og smaker på hjernemassen, 
kunne man umulig smake det samme som den som spiste sjokoladen fordi opplevelsen den personen med sjokoladen hadde er unik fordi den ikke bare kan komme av tilstander fra hjernen, 
men heller noe utenfor kroppen din \autocite[32-34]{Nagel2003}. 

Nagels ideer om bevissthet minner en del om Descartes' argumenter for dualisme.
\textit{Dualisme} handler nettopp om det Nagel framlegger; 
skillet mellom det fysiske, det som eksisterer i vår verden, og det mentale, det som eksisterer i vår bevissthet. 
Eller som Descartes kaller det, det \textit{objektive} og det \textit{subjektive} \autocite[156]{Dybvig2003}.


\subsection{Argumenter for en fysisk teori om bevissthet}


Nagel forklarer først at detaljene for fysisk teori om bevissthet er <<\emph{opp til vitenskapen å avdekke}>> som i seg selv veldig riktig.
For å kunne forklare en fysisk teori om bevissthet trengs det at det forskes på,
og undersøkes mer om nevrologi og menneskehjernen. Dette er da ikke lengre er et rent filosofisk spørsmål men også et vitenskapelig.

Senere rent benekter Nagel at fysiske hendelser i hjernen kan utgjøre en smaksopplevelse.
\begin{pquotation}{\cite[36]{Nagel2003}}
    ``Det er ikke mulig at et stort antall fysiske hendelser i hjernen,
uansett hvor kompliserte de måtte være, 
kan være de komponentene som utgjør smaksopplevelsene våre.''
\end{pquotation}
Begrunnelsen Nagel bruker når han kommer med denne påstanden er:
\begin{pquotation}{\cite[36]{Nagel2003}}
``\textit{vi [må] analysere noe mentalt - ikke noen ytre, observerbar fysisk substans (...) 
Et fysisk hele kan analyseres i mindre fysiske deler, men en
mental prosess kan ikke det, og fysiske deler kan ikke utgjøre et mentalt hele.}''
\end{pquotation}
Begrunnelsen til Nagel er altså at en fysisk teori om bevissthet ikke kan eksistere fordi bevisstheten, 
det mentale, er adskilt fra det fysiske. Rettere sagt, den er dualistisk. 
Her forkaster Nagel tanken om at bevisst kan være fysisk i bunn og grunn på hans egen oppfatning om at den er dualistisk.

Tankeeksperimentet hans kunne også bli brukt til å forklare f.eks. en identitetsteori.
\textit{Identitetsteori} går ut på at en fysisk prosess er identisk med en mental tilstand \autocite{snl:identitetsteori}.
Slik f.eks. en datamaskin kan utføre den samme operasjonen,
hvis den får repetert den samme kommandoen kontinuerlig.

Her kan vi også se at Nagels eksempel med sjokoladebiten i utgangspunktet 
ikke gir oss svar på så mye.
Vi kan like enkelt tenke oss at når man spiser denne sjokoladebiten,
er det fysiske tilstander i hjernen som produserer opplevelsen av smaken. 
Man kan da slikke så mye man vil på hjernen til den som spiser sjokoladebiten 
uten å smake sjokoladen, fordi smaken dannes av hjernen gjennom sanseinntrykk fra omverdenen.
Spiser man en ny bit av sjokoladen vil den smake likt som den forrige fordi hjernen 
utfører de samme fysiske tilstandene.
Denne identitetsteorien sier dermed at en mental tilstand er ikke noe mer enn en nevral tilstand,
som kan sammenliknes med at en tilstand i en datamaskin ikke noe mer enn en binær tallstreng.

Funksjonalisme 

Man kan derfor på grunnlag av identitetsteorien og funksjonalismen si at 
Thomas Nagels påstand om at argumentene mot en fysisk teori om bevissthet ikke holdet til.