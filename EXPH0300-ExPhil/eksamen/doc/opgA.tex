\section{Bevisthetsfilosofi}

\subsection{Hva er bevissthet?}

Bevissthet menes å være, i følge flere filosofer som f.eks. David Chalmers, et av de, hvis ikke det, 
vanskeligste problemet vitenskapen har å bryne seg på \autocite{Chalmers1995}. 
En måte å forklare hva bevisthet er kan være å se på forholdet mellom bevisstheten og hjernen.
Bevisstheten er avhengig av inntrykk den får fra hjernen enten direkte eller i form av sanser fra omverdenen. 
Bevisstheten er derfor opplevelsen man får av å sanse, tenke eller føle. 

Det er blant annen mange av disse argumentene Thomas Nagel bruker i sin te



Nagels ideer om bevissthet minner en del om Descartes' argumenter for dualisme \autocite[156]{Dybvig2003}.


\subsection{Argumenter for en fysisk teori om bevissthet}


Nagel forklarer først at detaljene for fysikalistisk teori om bevissthet er ``\emph{opp til vitenskapen å avdekke}'',
senere rent benekter han at fysiske hendelser i hjernen kan utgjøre en smaksopplevelse.
\begin{pquotation}{\cite[36]{Nagel2003}}
    ``Det er ikke mulig at et stort antall fysiske hendelser i hjernen,
uansett hvor kompliserte de måtte være, 
kan være de komponentene som utgjør smaksopplevelsene våre.''
\end{pquotation}

Hei dette er en test \autocite{snl:Böhmen} for å se om det funker

Nevral identitetsteori - en mental tilstand er ikke noe mer enn en nevral tilstant.
På lik linje med at en tilstand i en datamaskin ikke noe mer enn en binær tallstreng.