\section{Kan maskiner tenke?}

\subsection{Hva vil det si å kunne tenke?}

Hva betyr det å kunne tenke?
René Descartes mente at det å tenke var det som definerte at han faktisk eksisterte, <<\textit{Je pense, donc je suis}>> \autocite[30]{Descartes1637}.
For han var tanken helt grunnleggende, 
han hadde overbevist seg selv til å betvile alt annet, han kunne ikke vite om det virkelig eksisterte. 
Det eneste han visste var at han kunne tvile, tvilen er en tanke, derfor måtte han tenke for å kunne eksistere.

Det store spørsmålet blir da, er vi de eneste som kan tenke? 
Er vi alene om å være bevisst om at det eksisterer? 
Dette er store spørsmål som kan være svært vanskelige å besvare. 
Denne besvarelsen skal kun ta for seg en liten del av det første spørsmålet, 
kan maskiner tenke?

\subsection{Imitasjonsleken}

For å kunne si at en maskin kan tenke, må den kunne greie å overbevise mennesker til å tro at den kan tenke. Dette var konklusjonen til matematikeren Alan Turing da han tok for seg spørsmålet <<\textit{Kan maskiner tenke?}>> i 1950 \autocite{Kiran2013}. 
Turing mente at selve spørsmålet i seg selv var meningsløst, for Turing var det mer interessant å finne ut som en maskin kunne, når man observerte isolert utenfra, imitere et menneske eller greie å overbevise andre mennesker til å tro at den var et menneske.

Som resultat kom han fram til en test, en imitasjonslek, som gikk ut på at maskinen skulle prøve å fremstå som en menneske.
Kort fortalt gikk den ut på at i et rom satt en mann og en dame og i et annet rom satt en undersøker. 
Undersøkeren kunne kommunisere med de skriftlig, og oppgaven var å finne ut hvem som var mannen og hvem som var kvinnen.
Det Turing lurte på da var hva som ville skje hvis mannen eller kvinnen ble byttet ut med en maskin. 
Ville maskinen være i stand til å imitere mennesket?
\autocite[433-434]{Turing1950}.

Men vil en slik test som Turing legger frem kunne fungere? Turing anslo selv at innen 50 år ville datamaskinene ha nok minne til at et slik program kunne lages,
og at den gjennomsnittlige undersøkeren etter fem minutter ikke klare å peke ut maskinen i mer en 70 \% av tilfellene \autocite[442]{Turing1950}. 
Dessverre for Turing hadde han nok bare rett i estimatet om datamaskinens minne, for etter 71 år fra han kom med antydningen har den enda ikke vært et tilfelle der testen har blitt bestått etter hans kriterier \autocites[5]{Kiran2013}[]{turingTest}.

\subsection{Hva skiller maskiner fra mennesker?}

Et viktig spørsmål å stille seg er om turing-testen i utgangspunktet er en god test. 
Den svarer nemlig ikke på det som var det opprinnelige spørsmålet, kan maskiner tenke? 
Som sagt mente Turing spørsmålet var meningsløst, men mye har endret seg på 70 år. 
I dagens samfunn er kunstig intelligens mye utbredt, og i senere år har selvlærende maskiner blitt veldig populært.
En selvlærende maskin, eller maskinlæring, er i bunn og grunn en maskin eller et program som ikke er forhåndsprogrammert til å utføre en bestemt oppgave, 
men som gjennom å prosessere store mengder med data <<lærer>> seg hva den skal gjøre \autocite[3]{Kiran2013}. Men å snakke om læring i denne sammenhengen er litt missvisende, ihvertfall hvis vi sammenlikner med slik vi mennesker lærer. 

Når et menneske skal lære noe nytt, for eksempel en ny ferdighet, et nytt språk eller noe liknende, er man bevisste i handlingen man tar. 
Man tar selv initiativet til å lære og bevisst på eget ferdighetsnivå og eventuelle forkunnskaper, en maskin er ikke bevisst på samme måten. 
Når en selvlærende maskin skal lære noe nytt, for eksempel å gjenkjenne et menneske i et bilde, tar den ikke selv initiativet til å finne ut hva den må gjøre for å sette igang. 
Det er et menneske som må lage programmet som skal sette i gang maskinlærings prosessen, 
og som må finne fram testdataen som skal brukes til å trene opp maskinen. 
Maskinen er ikke selv bevisst på hva den skal gjøre slik mennesket er. 
Dette er et vesentlig skille mellom maskinen og mennesket, 
og en slik beskrivelse likner det den amerikanske filosofen John Searle kaller <<\textit{weak AI}>> eller svak KI på norsk \autocite[9]{Kiran2013}.

Et annet spørsmål en kan stille seg er da om det å være bevisst i det hele tatt er vesentlig når man lærer. 
Det høres kanskje absurd ut å ikke være bevisst om egen læring, men mange har muligens erfart å lære noe uten at man gikk inn for det. 
Ta for eksempel noe så grunnleggende som å gå.
De aller fleste mennesker uten noe form for funksjonell nedsettelse kan gå. 
Vi lærer det som regel ganske tidlig i livet, og før det er krabbing den eneste formen for bevegelse vi kan. 
Når et barn lærer seg å gå, er det som regel ikke det selv som har tenkt <<\textit{idag skal jeg lære meg å stå på to ben å gå}>>, det er ofte en eller begge av foreldrene som lærer barnet hva det skal gjøre. 
De løfter barnet på to ben og holder det i hendene mens det febrilsk forsøker å ta sine første skritt. 
Når foreldrene ikke aktivt lærer barnet, sitter det og observerer hvordan foreldrene går og beveger seg. 
Dette er litt på samme måte en maskin kan lære seg noe.
Den blir implisitt fortalt hva den skal gjøre uten at den egentlig er bevisst over hva det er, og blir <<holdt i hånden>> slik som barnet. 
Den blir fortalt hva som er forventet resultat og hva som er forventet å gjøre, slik som når barnet observerer hvordan foreldrene går. Man kan derfor heller si at man muligens ikke må være bevisst for å lære, men at man heller er avhengige av noen som er bevisste som kan fortelle deg hvordan du skal begynne å lære. 

