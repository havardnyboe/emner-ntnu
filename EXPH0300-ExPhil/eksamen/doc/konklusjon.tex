\section{Konklusjon}

Så for å konkludere, hva har besvarelsen kommet fram til? 
Kort sagt har den kommet fram til at bevisstheten er et stort og vanskelig tema å definere, og at det er veldig relevant når man skal prøve å finne svar på spørsmål som <<\textit{kan maskiner tenke?}>>. 
Den har også kommet fram til at Thomas Nagels sjokolade-tankeeksperiment ikke er det beste når man skal prøve å avvise argumentene for en fysisk teori om bevissthet.

Videre har den kommet fram til at problemstillingen om å finne ut om maskiner kan tenke kan angripes fra to standpunkter. Man kan enten forsøke å finne ut om maskinene kan imitere mennesker slik som Alan Turing forsøkte, eller man kan prøve å sammenlikne måten mennesker lærer med måten maskiner lærer. 
Den har også kommet fram til at maskiner ikke er bevisste på samme måte som oss mennesker, men at det ikke nødvendigvis er et hinder for at maskiner kan lære slik som mennesker lærer.
Likevel sitter vi her enda og diskuterer rot-spørsmålet <<\textit{kan maskiner tenke?}>>, som for Turing i 1950 ville være utenkelig. 
\vspace{-1em}
\begin{pquotation}{\cite[442]{Turing1950}}
    ``Nevertheless I believe that at the
    end of the century the use of words and general educated opinion will have
    altered so much that one will be able to speak of machines thinking without
    expecting to be contradicted''
\end{pquotation}
