% #region PREAMBEL OG PAKKER
\documentclass[a4paper, 12pt]{article}  % DOKUMENTKLASSE
\title{Øving 3}                         % TITTEL
\author{Håvard Solberg Nybøe}           % FORFATTER
\date{\today}                           % DATO & FAG

\usepackage[english, norsk]{babel}      % NORSK SPRÅK
\usepackage[
    backend=biber,style=ieee]{biblatex} % BIBLIOGRAFI
\usepackage{csquotes}                   % PAKKE TIL BABEL
\addbibresource{bibliografi.bib}        % PATH TIL BIBLIOGRAFI
\usepackage[hidelinks]{hyperref}        % LENKER I TOC OG GENERELT
\usepackage[margin=1in]{geometry}       % VANLIG STØRRELSE MARGIN
\setlength{\parindent}{0em}             % SKILLER AVSNITT
\setlength{\parskip}{.8em}              % SKILLER AVSNITT
\usepackage{graphicx}                   % BILDER \includegraphics[OPTIONS]{PATH}
\usepackage{kantlipsum}                 % FYLLTEKST I KANT-STIL (kant[n-m])
\usepackage{amsfonts,                   % BLACKBOARD BOLD FONT (\mathbb{N})
amsmath,stmaryrd,amssymb}               % ANDRE MATTE PAKKER
\usepackage{circuitikz}                 % LOGISKE PORTER OG KRETSER & TikZ
\usepackage{import}                     % IMPORTER FILER (\import{PATH}{FILE})
\usepackage{caption}                    % PAKKE FOR BEDRE CAPTIONS I FIGURER
\usepackage{float}                      % FLYTT FIGURER 
% #endregion
\begin{document}

\maketitle
% \tableofcontents % INNHOLDSFORTEGNELSE

\section*{Oppgave 1}
% Forklar bakgrunn for og innhold av begrepet «affordance»
Begrepet <<affordance>> ble introdusert av den amerikanske psykologen James J. Gibson på 1970-tallet. 
Han brukte det i sammenheng med hans teori om persepsjon og økologisk psykologi. Affordance refererer til de potensielle handlingene eller bruksmulighetene som et objekt eller miljø tilbyr til en organisme \cite{volkoff2017}.
På norsk kan man gjerne bruke begrepet \emph{tilbydelighet} om affordance \cite{borgund2019}.

For å forklare det enkelt, handler affordance om egenskaper ved et objekt eller en situasjon som indikerer hvordan det kan eller skal brukes.
Det er en relasjon mellom den observerende brukeren og den observerte gjenstanden eller miljøet.
Affordance er ikke bare knyttet til de fysiske egenskapene til objekter, men også til hvordan disse egenskapene blir oppfattet av brukeren.

Eksempelvis kan man se for seg en knapp på en nettside.
Denne knappen har en fysisk egenskap som gjør at den kan trykkes på.
Den har også en visuell egenskap som gjør at den er utformet slik at den skiller seg ut fra resten av nettsiden. 
Knappen tilbyr dermed en handling til brukeren, nemlig å trykke på den.
Dette er affordance.

Affordance er et viktig prinsipp innen design, spesielt i interaksjonsdesign og brukergrensesnitt, der det brukes til å beskrive hvordan designelementer kommuniserer bruksmuligheter til brukeren \cite{volkoff2017}.
Det er også et viktig prinsipp innenfor arkitektur, der det brukes til å beskrive hvordan bygninger og rom kommuniserer bruksmuligheter til brukeren.


\section*{Oppgave 2}
% Diskuter oppslaget fra DigDir (Direktoratet for Digitalisering) der de presenterer et case om deling av data. Trekk inn relevante perspektiv fra kurset.

% Innledning
Oppslaget fra DigDir git innsikt i utviklingen av en digital løsning for å søke om økonomisk sosialhjelp gjennom et samarbeid mellom NAV og og Husbanken.
Prosjektet, som går under navnet Digisos, har som mål å effektivisere og forenkle søknadsprosessen ved å innhente hvor mye brukerne har fått i bostøtte fra Husbanken og integrere dette i søknaden \cite{digisos}.
Oppslaget tar for seg ulike aspekter av prosjektet, inkludert teknologiske utfordringer, juridiske avklaringer, og fokuset på brukeropplevelse.

% Teori
I lys av teori kan vi trekke inn relevante prinsipper som dataminimering og personvern. 
Dataminimering vil si  å samle inn og behandle så lite personlig informasjon som nødvendig.
I konteksten av Digisos-prosjektet innebærer dataminimering å være svært selektiv i hvilken informasjon som blir hentet fra Husbanken for å støtte søknadsprosessen.
Dette hindrer innsamling av unødvendig eller overflødig data, og det sikrer at bare de dataene som er avgjørende for å behandle søknader om økonomisk sosialhjelp, blir samlet inn.
Dette er essensielt for å beskytte brukernes personvern og sikre at prosjektet ikke går utover det som er nødvendig for sin hensikt.

Tverrfaglig samarbeid er et nøkkelprinsipp som understrekes i prosjektet, da det kreves komplementære ferdigheter innen teknologi, jus og forretningsforståelse for å oppnå suksess.
Alle disse fagområdene må jobbe sammen for å forstå og håndtere hverandres utfordringer. 
For eksempel krever implementeringen av datadeling-systemet at utviklere samarbeider tett med jurister for å sikre at all datainnsamling og utveksling er i samsvar med gjeldende lovverk. 
Samtidig må perspektivene om brukeropplevelse og brukervennlighet forstås for å sikre at løsningen møter brukernes behov og forventninger.

% Diskusjon
Prosjektet står overfor ikke ubetydelige utfordringer, særlig knyttet til de juridiske avklaringene, og de teknologiske integrasjonene. 
Når det gjelder juridiske aspekter, er en av de sentrale utfordringene behovet for å sikre at all datadeling følger gjeldene personvernlover.
Dette krever nøye juridiske vurderinger og dialog med relevante myndigheter for å lage tydelige retningslinjer for hvordan data kan samles inn og deles mellom de ulike aktørene.

I tillegg er det viktig å følge med på endringer i personvernlover over tid. 
Prosjektet må være fleksibelt nok til å tilpasse seg nye krav og standarder som kan dukke opp. 
Dette krever kontinuerlig juridisk oppfølging gjennom hele prosjektet.

Når det gjelder teknologisk integrasjon, er en sentral utfordring utviklingen av funksjonelle API-er for å muliggjøre sikker og effektiv datautveksling med Husbankens systemer. 
Dette krever et tett samarbeid mellom utviklere i prosjektet og Husbanken for å utforme grensesnitt som ikke bare oppfyller tekniske standarder, men også sikrer at sensitiv informasjon blir forsvarlig håndtert.

Teknologisk kompleksitet kommer også til syne når man ser på hvordan den digitale løsningen integreres med eksisterende systemer i NAV. 
Dette krever tilpasninger og testing for å sikre en sikker og pålitelig datastrøm mellom de ulike aktørene.
Eventuelle feil i denne integrasjonen kan potensielt få konsekvenser for hele søknadsprosessen, noe som understreker viktigheten av grundig teknisk testing og oppfølging.

% Konklusjon
Digisos-prosjektet demonstrerer vellykket bruk av prinsippene dataminimering, tverrfaglig samarbeid og juridiske avklaringer i utviklingen av en digital løsning for økonomisk sosialhjelp.
Gjennom nøye begrensning av informasjonssamling, samarbeid mellom ulike fagfelt og grundig juridisk håndtering, har prosjektet oppnådd en balansert tilnærming.
Dette understreker betydningen av slike prinsipper for vellykket digital transformasjon i offentlig sektor, der brukerens personvern og lovlighet er sentrale faktorer for suksess.

\printbibliography[heading=bibintoc] % LAGER BIBLIOGRAFI
\end{document}
