% #region PREAMBEL OG PAKKER
\documentclass[a4paper, 12pt]{article}  % DOKUMENTKLASSE
\title{Øving 2}               % TITTEL
\author{Håvard Solberg Nybøe}           % FORFATTER
\date{\today}                    % DATO & FAG

\usepackage[english, norsk]{babel}      % NORSK SPRÅK
\usepackage[
    backend=biber,style=ieee]{biblatex}  % BIBLIOGRAFI
\usepackage{csquotes}                   % PAKKE TIL BABEL
\addbibresource{bibliografi.bib}        % PATH TIL BIBLIOGRAFI
\usepackage[hidelinks]{hyperref}        % LENKER I TOC OG GENERELT
\usepackage[margin=1in]{geometry}       % VANLIG STØRRELSE MARGIN
\setlength{\parindent}{0em}             % SKILLER AVSNITT
\setlength{\parskip}{.8em}              % SKILLER AVSNITT
\usepackage{graphicx}                   % BILDER \includegraphics[OPTIONS]{PATH}
\usepackage{kantlipsum}                 % FYLLTEKST I KANT-STIL (kant[n-m])
\usepackage{amsfonts,                   % BLACKBOARD BOLD FONT (\mathbb{N})
amsmath,stmaryrd,amssymb}         % ANDRE MATTE PAKKER
\usepackage{circuitikz}                 % LOGISKE PORTER OG KRETSER & TikZ
\usepackage{import}                     % IMPORTER FILER (\import{PATH}{FILE})
\usepackage{caption}                    % PAKKE FOR BEDRE CAPTIONS I FIGURER
\usepackage{float}                      % FLYTT FIGURER 
% #endregion
\begin{document}

\maketitle
% \tableofcontents % INNHOLDSFORTEGNELSE

\section*{Oppgave 1}
% Beskriv hovedtrekkene teknologi-synene teknologisk determinisme og sosial konstruktivisme.
% Hvilke hovedutfordringer har de to?

Teknologisk determinisme handler om at teknologi i stor grad bestemmer utviklingen av samfunnet og dets strukturer. 
Innovasjoner og framskritt i teknologi er det som driver samfunnet framover. \parencite{balterzen08}
Teknologisk determinisme er en teori som har blitt brukt til å forklare hvordan teknologi har påvirket samfunnet, og hvordan samfunnet blir påvirket av teknologi.
Teknologisk determinisme slik Andrew Feenberg definerer det består av to teser:
\begin{itemize}
    \item Tese 1: Utviklingen av teknologi er bestemt på forhånd og påvirkes ikke av ytre faktorer.
    \item Tese 2: Teknologien er en ytre faktor sammenliknet med andre samfunnsfaktorer, og utgjør en bestemmende påvirkning på samfunnet.
\end{itemize}

De fleste varianter av teknologisk determinisme baserer seg på disse to idéene. Tese 1 antyder at det finnes en uavhengig utvikling av teknologi, mens tese 2 fremhever teknologien som en kraft utenfra som utøver en direkte innflytelse på samfunnet, uavhengig av sosiale faktorer. \parencite{balterzen08}

Hovedutfordringen med teknologisk determinisme er at det ikke tar hensyn til at teknologi er et produkt av samfunnet. 
Mennesker kan fraskrive seg ansvaret for teknologien de skaper, og late som om det ikke er en del av deres ansvar å ta hensyn til hvordan teknologien påvirker samfunnet.
Eksempelvis slik teknologiselskaper som Facebook og TikTok gjør når de forsøker å fraskrive seg ansvaret for hvordan deres teknologi er laget for å være avhengighetsskapende, og hvordan det påvirke brukerne til å bruke mer tid på plattformen.

Sosialkonstruktivisme er en teori som handler om at teknologi er et produkt av samfunnet, og at teknologi er et resultat av sosiale prosesser.
Sammenliknet med teknologisk determinisme, som mener at teknologi er en uavhengig faktor som påvirker samfunnet, mener sosialkonstruktivisme at teknologi er et produkt av samfunnet, og at teknologi er et resultat gjensidig påvirkning mellom teknologi og samfunn.\parencite{scot}
I sosialkonstruktivismen sees verdien av teknologi i kontekst av at teknologien er ladet med interesser, og ikke en nøytral faktor som beslutter hvordan samfunnet skal utvikle seg.

Hovedutfordringene med sosialkonstruktivisme er at det involverer flere aktør-grupper som har ulike interesser, og at det kan være vanskelig å komme til enighet om hvordan teknologien skal utvikles. 
Det er også en uklar grense mellom det sosiale og det teknologiske, og hvordan begge disse aspektene konstrueres gjennom sosiale prosesser.
Imidlertid kan det være vanskelig å trekke klare grenser mellom dem, da teknologi ofte er integrert i sosiale strukturer og praksiser.


\section*{Oppgave 2}
% Innledning - teori - diskusjon - konklusjon
% Diskuter oppslaget om Decidim i lenken under i lys av perspektiv på medvirkning, men også teknologi-syn

% Innledning
Decidim står fram som en ny digital plattform som gir stemme til fellesskapet og styrker deltakende demokrati.
Prosjektet er et samarbeid mellom kommunene i Smartbyene og innovasjonsavdelingen i KS. 
Navnet Decidim, som betyr <<Vi bestemmer>> på katalansk, viser at denne løsningen er viktig for å forandre hvordan beslutninger blir tatt.
Denne teksten tar sikte på å belyse Decidim gjennom perspektiver på medvirkning, hvor fellesskapets stemme står i sentrum, samt teknologisyn, som ser på hvordan plattformen ikke bare er et verktøy, men en dynamisk kraft i demokratiske prosesser.

% Teori
% Medvirkning
Decidim er fundamentert i teorien om deltakende demokrati, hvor målet er å demokratisere beslutningsprosesser ved å gi fellesskapet direkte innflytelse.
Plattformen legger vekt på inklusivitet, åpne diskusjoner, og avstemninger for å sikre at alle parter har mulighet til å bidra til beslutningsprosesser.
Den omfavner teoretiske prinsipper om deltakelse som en nøkkelkomponent i demokratiske systemer.

% Teknologisyn
Decidim adopterer en tilnærming som går utover konvensjonell teknologisyn. 
Ved å være basert på åpen kildekode reflekterer Decidim ideen om samarbeid, tilpasning og gjenbruk av teknologiske løsninger.
Dette teknologisynet utfordrer også deterministiske perspektiver ved å vektlegge den gjensidige formingen av teknologi og samfunn, hvor brukernes interaksjoner og samfunnsbehov påvirker teknologiens utvikling.

% Diskusjon
% Medvirkning
Decidim muliggjør en grad av medvirkning som går utover tradisjonelle former for deltakelse. Det åpner kanaler for at borgere kan være med på å forme beslutninger som påvirker deres liv direkte. Samtidig kan utfordringer oppstå, som for eksempel representativitet og spørsmål om digital deltakelse er tilgjengelig for alle befolkningsgrupper.

% Teknologisyn
Åpen kildekode-tilnærmingen i Decidim reflekterer et syn på teknologi som en felles ressurs for samfunnet. Denne tilnærmingen gir rom for innovasjon og tilpasning, men det krever også at brukere har en viss teknologisk kompetanse. Det å se teknologi som gjensidig formende, utfordrer det ensidige synet på teknologisk determinisme, og understreker at teknologi er formet av og former samfunnet.

% Konklusjon
Decidim står som en modell for fremtiden, hvor medvirkning og teknologisyn er tett sammenvevd. Det utfordrer tradisjonelle maktstrukturer ved å gi fellesskapet en direkte rolle i beslutningsprosesser. Likevel har Decidim sine utfordringer, og diskusjoner om inklusivitet og tilgjengelighet må være en del av veien videre. Samtidig viser det hvordan teknologi ikke bare er en mekanisme, men en aktiv del i formingen av demokratiske praksiser. Decidim peker mot et demokratisk landskap hvor skillet mellom teknologi og medvirkning smelter sammen for å skape en mer deltakende fremtid.

\printbibliography[heading=bibintoc] % LAGER BIBLIOGRAFI
\end{document}
