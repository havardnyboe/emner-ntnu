% #region PREAMBEL OG PAKKER
\documentclass[a4paper, 12pt]{article}  % DOKUMENTKLASSE
\title{Øving 1}                         % TITTEL
\author{Håvard Solberg Nybøe}           % FORFATTER
\date{\today}                           % DATO & FAG

\usepackage[english, norsk]{babel}      % NORSK SPRÅK
\usepackage[
    backend=biber,style=apa]{biblatex}  % BIBLIOGRAFI
\usepackage{csquotes}                   % PAKKE TIL BABEL
\addbibresource{bibliografi.bib}        % PATH TIL BIBLIOGRAFI
\usepackage[hidelinks]{hyperref}        % LENKER I TOC OG GENERELT
\usepackage[margin=25mm]{geometry}       % VANLIG STØRRELSE MARGIN
\setlength{\parindent}{0em}             % SKILLER AVSNITT
\setlength{\parskip}{.8em}              % SKILLER AVSNITT
\usepackage{graphicx}                   % BILDER \includegraphics[OPTIONS]{PATH}
\usepackage{kantlipsum}                 % FYLLTEKST I KANT-STIL (kant[n-m])
\usepackage{amsfonts,                   % BLACKBOARD BOLD FONT (\mathbb{N})
amsmath,stmaryrd,amssymb}         % ANDRE MATTE PAKKER
\usepackage{circuitikz}                 % LOGISKE PORTER OG KRETSER & TikZ
\usepackage{import}                     % IMPORTER FILER (\import{PATH}{FILE})
\usepackage{caption}                    % PAKKE FOR BEDRE CAPTIONS I FIGURER
\usepackage{float}                      % FLYTT FIGURER 
% #endregion
\begin{document}

\maketitle

\section*{Oppgave 1}

% Lineær innovasjon:
%   - Slide 10 (Foreles 3)
%   - Grunnforskning -> Anvendt forskning -> Teknisk utvikling -> Industriell tilpassing -> Spredning
%   - Lineær modell bugger på Schumpeter (Push, Økonomisk modell)
% Hovedutfordringer:
%   - Samdarbeid/læring ikke tatt (godt) høyde for
%   - Klynger/komplementaritet ikke tatt (godt) høyde for

Hovedtrekkene for tradisjonell lineær innovasjon følger gjerne denne rekken:
\parencite{sorensen2006}


\begin{center}
    Naturvitenskaplig grunnforskning
    \\$\downarrow$ 
    \\Teknologisk/anvendt forskning
    \\$\downarrow$
    \\Teknologisk utviklingsarbeid
    \\$\downarrow$
    \\Industriell tilpassing og utprøving
    \\$\downarrow$
    \\Spredning
\end{center}

Denne modellen baserer seg på at det først oppstår et utgangspunkt fra en grunnleggende forskning (Naturvitenskaplig grunnforskning), som videre anvendes i ny forskning (Teknologisk forskning). Resultatene fra denne forskningen gir springbrett for utprøving og prototyper (Teknologisk utviklingsarbeid), som videreutvikles og forbedres av bedrifter (Industriell tilpassing og utprøving), før det tilslutt spres til andre bedrifter og brukere.

Imidlertid er ikke denne modellen en gjenspeiling av slik realiteten som oftest er, og teknologisk innovasjon skjer i stor grad gjennom utprøving og grundervirksomhet, framfor anvendt forskning. \parencite{sorensen2006} 

Denne modellen har store likhetstrekk til den lineære fossefalls-metodikken for arbeid innen programvareutvikling, og deler også noen av hovedutfordringene sine med den. Blant annet er samarbeid noe som ikke er tatt særlig høyde for i denne modellen. Idéene og utviklingen ``dyttes'' fra fase til fase, og baseres på stor grad i at nestemann i rekken skal ta over og videreutvikle det som er gjort. Dette kan føre til at det ikke er en god nok forståelse av hva som er gjort, og det blir vanskeligere å bygge videre på arbeidet.


\section*{Oppgave 2}

I NRKs artikkel ``Lærere fortvilet over ny kunstig intelligens'' møter vi tre parter; Landslaget for norskundervisning, prosjektleder for forskning og kunstnerisk utviklingsarbeid på Høyskolen Kristiania Morten Irgens, og professor i nordisk litteratur Erik Vassenden. Disse tre deler en oppfatning av at inntoget av tjenester som ChatGPT skapet utfordringer for lærere i skolen, men har til dels litt ulik oppfatning av alvorlighetsgraden, og hvor store innvirkninger dette vil ha.

I starten av artikkelen møter vi Landslaget for norskundervisning og uttalelsen de har levert til stortingspolitikerne. I brevet beskriver de en frykt for hva disse verktøyene kan gjøre med elevenes skrive- og leseferdigheter, og at det på sikt vil true demokratiet og ny idé- og kunnskapsutvikling. Uttalelsen Landslaget for norskundervisning kommer med her er en ganske deterministisk tankegang om at disse verktøyene vil føre til at elevene ikke lenger vil lære seg å skrive, og at de vil miste evnen til å tenke kritisk.

I artikkelen kommer det frem at Irgens og Vassenden både forstår og deler litt av frykten til lærerne, men de har ulike framgangsmåter for hvordan håndteringen av denne utfordringen skal gjennomføres. Når Vassenden får spørsmålet: ``\textit{Tror du at du vil klare å avsløre juks, hvis noen bruker den til å skrive norskoppgaver for seg?}'', svarer han ``\textit{Jeg tror det krever at vi lager oppgaver som er egnet til å avsløre slikt juks (...) vi [bør] unngå oppgaver som inviterer til passiv montering av informasjon.}'' Vassenden tar til ordet for at man viss grad burde forsøke å tilpasse undervisningen slik at den ikke kommer i konflikt med, eller kan bli utnyttet av slike verktøy; at å tilpasse seg til dem, er bedre enn å forsøke å ``bekjempe'' eller unngå dem. Dette perspektivet kan sies å tilhøre en sosialkonstruktivistisk holding, der intuisjonene, her skolen, må tilpasse seg til den samfunnsmessige utviklingen.

Irgens deler mye av det samme perspektivet som Vassenden ut uttaler sier blant annet: ``\textit{Det er mange som kommer til å rive seg i håret, men jeg gjør ikke det. Som med all teknologi tilpasser vi oss. (...) studentene [vil] bruke chatboter og andre avanserte verktøy på samme måte som man har brukt biblioteket.}''

Landslaget for norskundervisning er derimot ikke like tilpasningsdyktige og har to framtidsscenarier de ser for seg. Det ene er å gå tilbake til penn og papir i stor grad, og det andre er å sørge for teknologi og ressurser som hindrer bruken av slike ai-verktøy. Begge scenariene faller egentlig ut av det samme, nemlig å hindre tilgangen og bruken av slike verktøy. Her ser vi igjen en svært deterministisk tankerekke, der det å tilpasse seg og omfavne denne nye teknologien ikke er et alternativ.

\printbibliography[heading=bibintoc] % LAGER BIBLIOGRAFI
\end{document}
