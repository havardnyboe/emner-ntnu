\section{Hva er parprogrammering?}

% presentere teori og definisjoner av parprogrammering
% presentere min definisjon av parprogrammering
% 

Parprogrammering er en metode innen ekstremprogrammering der to utviklere jobber sammen. Den ene utvikleren, <<sjåføren>>, skriver koden mens den andre, <<navigatøren>>, følger med, leser koden, og gir tilbakemeldinger underveis. Etter en bestemt tid kan utviklere bytte roller \cite{McDowell2006}. 
I en meta-analyse gjort av Hannay, Dybå, Arisholm og Sjøberg i 2009 kommer det frem at parprogrammering er et nyttig verktøy for å øke kvaliteten på komplekse oppgaver, og produktivitet på enklere oppgaver \cite{hannay2009}.
En annen studie, av McDowell et. al. i 2006 viser at parprogrammering har stor virkning som et pedagogisk verktøy, og at det er en god måte for studenter å lære programmering på \cite{McDowell2006}.
Parprogrammering ser også ut til å være nyttig for å ivareta og utvikle studenters interesse for programmering, og særlig blant kvinner, en stadig underrepresentert gruppe i IT-bransjen \cite{McDowell2006}\cite{kattan2018}.

Parprogrammering differensierer seg fra å bare <<kode-sammen>> ved at det er en strukturert måte å jobbe på; den ene skriver koden, mens den andre følger med og gir tilbakemeldinger \cite{kattan2018}. 
Dette er en viktig forskjell fra å bare kode sammen, hvor det er vanskeligere å både få, og gi, tilbakemeldinger på koden.
