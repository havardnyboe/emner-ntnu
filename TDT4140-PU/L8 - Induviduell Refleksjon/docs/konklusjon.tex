\section{Konklusjon}

For å svare på problemstillingen har parprogrammering påvirket samarbeidet i den forstand at gruppen raskere kom i gang med å jobbe sammen, og at vi fikk en bedre forståelse av hverandres arbeidsmetoder.
Parprogrammering har også påvirket produktiviteten slik at vi fikk mer gjort på kortere tid, og at vi fikk arbeidet mer sammen.
Som gruppe var vår største svakhet at vi ikke hadde en god nok forståelse av hva parprogrammering faktisk innebar, og derfor ikke fikk utnyttet metoden til det fulle potensialet.
Effekten parprogrammering hadde på produktiviteten var derfor ikke like stor som den kunne vært, sammenliknet med det enkelte studier har vist.
Vi har også sett at parprogrammering har påvirket kvaliteten på koden, og at vi har fått en bedre forståelse av hvordan vi kan skrive bedre kode.
