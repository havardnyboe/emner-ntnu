\section{Konklusjon}

For å svare på problemstillingen har parprogrammering påvirket samarbeidet i den forstand at gruppen raskere kom i gang med å jobbe sammen, og at vi fikk en bedre forståelse av hverandres arbeidsmetoder.
Parprogrammering har også påvirket produktiviteten i den forstand at vi fikk mer gjort på kortere tid, og at vi fikk arbeidet mer sammen.
Som gruppe var vår største svakhet at vi ikke hadde en god nok forståelse av hva parprogrammering faktisk innebar, og derfor ikke fikk utnyttet metoden til det fulle potensialet.

